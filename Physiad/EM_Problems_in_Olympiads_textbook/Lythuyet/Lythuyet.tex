\begin{appendices}
\chapter*{Lý thuyết điện từ}
\addcontentsline{toc}{chapter}{\bfseries Phụ lục.\quad Lý thuyết điện từ}
\section{Những kiến thức toán quan trọng.}
\subsection{Vector.}
\begin{align*}
    \ot{a}&=a_x\ot{i}+a_y\ot{j}+a_z\ot{k},\\
    \ot{b}&=b_x\ot{i}+b_y\ot{j}+b_z\ot{k}.
\end{align*}
Tích vô hướng:
\begin{align*}
    \ot{a}\cdot\ot{b}&=\left(a_x\ot{i}+a_y\ot{j}+a_z\ot{k}\right)\cdot\left(b_x\ot{i}+b_y\ot{j}+b_z\ot{k}\right)\\
    &= a_xb_x+a_yb_y+a_zb_z.
\end{align*}
Tích hữu hướng hai vector:
\begin{align*}
    \ot{a}\times\ot{b}&=\left(a_x\ot{i}+a_y\ot{j}+a_z\ot{k}\right)\times\left(b_x\ot{i}+b_y\ot{j}+b_z\ot{k}\right)\\
    &=\left(a_yb_z-a_zb_y\right)\ot{i}+\left(a_zb_x-a_xb_z\right)\ot{j}+\left(a_xb_y-a_yb_x\right)\ot{k}.
\end{align*}
Tích hữu hướng của hai vector có thể được viết dưới dạng định thức của ma trận:
$$\ot{a}\times\ot{b}=\left|\begin{array}{ccc}
\ot{i} & \ot{j} & \ot{k}\\
a_x & a_y & a_z\\
b_x & b_y & b_z
\end{array}\right|.$$
Tích của ba vector:
$$\ot{a}\cdot\left(\ot{b}\times\ot{c}\right)=\ot{b}\cdot\left(\ot{c}\times\ot{a}\right)=\ot{c}\cdot\left(\ot{a}\times\ot{b}\right),$$
hay có thể viết ở dạng thành phần
$$\ot{a}\cdot\left(\ot{b}\times\ot{c}\right)=\left|\begin{array}{ccc}
a_x & a_y & a_z\\
b_x & b_y & b_z\\
c_x & c_y & c_z\\
\end{array}\right|.$$
Tích hữu hướng của ba vector:
$$\ot{a}\times\left(\ot{b}\times\ot{c}\right)=\ot{b}\left(\ot{a}\cdot\ot{c}\right)-\ot{c}\left(\ot{a}\cdot\ot{b}\right).$$
\subsection{Các toán tử vector.}
Toán tử nabla trong toạ độ Descartes:
$$\ot{\nabla}=\ot{i}\dfrac{\partial}{\partial x}+\ot{j}\dfrac{\partial}{\partial y}+\ot{k}\dfrac{\partial}{\partial k}.$$
\subsubsection{Gradient.}
Khi $\phi(x,y,z)$ là một trường vô hướng, gradient của $\phi$ được định nghĩa bởi vector:
$$\dfrac{\partial \phi}{\partial x}\ot{i}+\dfrac{\partial \phi}{\partial y}\ot{j}+\dfrac{\partial \phi}{\partial z}\ot{k},$$
Ta có thể viết thành:
$$\mathrm{grad}\phi=\ot{\nabla} \phi=\left(\ot{i}\dfrac{\partial}{\partial x}+\ot{j}\dfrac{\partial}{\partial y}+\ot{k}\dfrac{\partial}{\partial k}\right)\phi,$$
trong đó $\ot{\nabla}=\ot{i}\dfrac{\partial}{\partial x}+\ot{j}\dfrac{\partial}{\partial y}+\ot{k}\dfrac{\partial}{\partial k}$ được gọi là toán tử nabla.\\
Tích vô hướng của $\ot{\nabla}\phi$ và vi phân của vector độ dời $\dd \ot{r}$:
\begin{align*}
    \ot{\nabla}\phi\cdot\dd \ot{r}&=\left(\dfrac{\partial \phi}{\partial x}\ot{i}+\dfrac{\partial \phi}{\partial y}\ot{j}+\dfrac{\partial \phi}{\partial z}\ot{k}\right)\cdot\left(\dd x\ot{i}+\dd y\ot{j}+\dd z\ot{k}\right)\\
    &=\dfrac{\partial \phi}{\partial x}\dd x+\dfrac{\partial \phi}{\partial y}\dd y+\dfrac{\partial \phi}{\partial z}\dd z\\
    &=\dd \phi.
\end{align*}
Tương tự như đạo hàm của một hàm hợp, toán tử gradient của một hàm hợp có thể được tính bằng công thức:
$$\ot{\nabla}\phi(\varphi)=\dfrac{\partial \phi}{\partial \varphi}\ot{\nabla}\varphi.$$
với $\phi$ và $\varphi$ đều là các trường vô hướng.
\subsubsection{Divergence.}
Divergence của một trường vector $\ot{a}(x,y,z)$ có thể được định nghĩa bởi công thức:
\begin{align*}
\mathrm{div} \ot{a}=\ot{\nabla}\cdot \ot{a}
&=\left(\ot{i}\dfrac{\partial}{\partial x}+\ot{j}\dfrac{\partial}{\partial y}+\ot{k}\dfrac{\partial}{\partial k}\right)\cdot \ot{a},\\
&=\dfrac{\partial a_x }{\partial x}+\dfrac{\partial a_y }{\partial y}+\dfrac{\partial a_z}{\partial z}.\\
\end{align*}
\subsubsection{Rotation.}
Với $\ot{a}(x,y,z)$ là một trường vector, rotation của $\ot{a}$ được định nghĩa bởi công thức:
\begin{align*}
    \text{rot}\ot{a}&=\ot{\nabla}\times\ot{a}=\left|\begin{array}{ccc}
    \ot{i} & \ot{j} & \ot{k} \\
    \dfrac{\partial}{\partial x} & \dfrac{\partial}{\partial y} & \dfrac{\partial}{\partial z} \\
    a_x & a_y & a_z 
    \end{array}\right|\\
    &=\left(\dfrac{\partial a_z}{\partial y}-\dfrac{\partial a_y}{\partial z}\right)\ot{i}+\left(\dfrac{\partial a_x}{\partial z}-\dfrac{\partial a_z}{\partial x}\right)\ot{j}+\left(\dfrac{\partial a_x}{\partial y}-\dfrac{\partial a_y}{\partial x}\right)\ot{i}.
\end{align*}
\subsubsection{Một số toán tử vector của tổng và tích vector.}
\begin{align*}
    \ot{\nabla}(\phi+\varphi)&=\ot{\nabla}\phi+\ot{\nabla}\varphi,\\
    \ot{\nabla}\cdot\left(\ot{a}+\ot{b}\right)&=
    \ot{\nabla}\cdot\ot{a}+\ot{\nabla}\cdot\ot{b},\\
    \ot{\nabla}\times\left(\ot{a}+\ot{b}\right)&=\ot{\nabla}\times\ot{a}+\ot{\nabla}\times\ot{b},\\
    \ot{\nabla}(\phi\varphi)&=\varphi\ot{\nabla}\phi+\phi\ot{\nabla}\varphi,\\
    \ot{\nabla}(\ot{a}\cdot\ot{b})&=\ot{a}\times(\ot{\nabla}\times\ot{b})+\ot{b}\times(\ot{\nabla}\times\ot{a})+(\ot{a}\cdot\ot{\nabla})\ot{b}+(\ot{b}\cdot\ot{\nabla})\ot{a},\\
    \ot{\nabla}\left(\phi\ot{a}\right)&=\phi\ot{\nabla}\cdot\ot{a}+\ot{a}\cdot\ot{\nabla}\phi,\\
    \ot{\nabla}\cdot(\ot{a}\times\ot{b})&=\ot{b}\cdot\left(\ot{\nabla}\times\ot{a}\right)-\ot{a}\cdot(\ot{\nabla}\times\ot{b}),\\
    \ot{\nabla}\times\left(\phi\ot{a}\right)&=\ot{\nabla}\phi\times\ot{a}+\phi\ot{\nabla}\times\ot{a},\\
    \ot{\nabla}\times(\ot{a}\times\ot{b})&=\ot{a}(\ot{\nabla}\cdot\ot{b})-\ot{b}(\ot{\nabla}\cdot\ot{a})+(\ot{b}\cdot\ot{\nabla})\ot{a}-(\ot{a}\cdot\ot{\nabla})\ot{b}.
\end{align*}
với $\phi$, $\varphi$ là các trường vô hướng, $\ot{a}$ và $\ot{b}$ là các trường vector.
\subsection{Các hệ toạ độ.}
\subsubsection{Hệ toạ độ trụ.}
\begin{center}


\tikzset{every picture/.style={line width=0.75pt}} %set default line width to 0.75pt        

\begin{tikzpicture}[x=0.75pt,y=0.75pt,yscale=-1,xscale=1]
%uncomment if require: \path (0,438); %set diagram left start at 0, and has height of 438

%Straight Lines [id:da6020863454544023] 
\draw    (280.4,260.32) -- (556.8,260.6) ;
\draw [shift={(558.8,260.6)}, rotate = 180.06] [color={rgb, 255:red, 0; green, 0; blue, 0 }  ][line width=0.75]    (10.93,-3.29) .. controls (6.95,-1.4) and (3.31,-0.3) .. (0,0) .. controls (3.31,0.3) and (6.95,1.4) .. (10.93,3.29)   ;
%Straight Lines [id:da2807572666677913] 
\draw    (280.4,260.32) -- (270.21,266.17) -- (92.53,369.26) ;
\draw [shift={(90.8,370.27)}, rotate = 329.88] [color={rgb, 255:red, 0; green, 0; blue, 0 }  ][line width=0.75]    (10.93,-3.29) .. controls (6.95,-1.4) and (3.31,-0.3) .. (0,0) .. controls (3.31,0.3) and (6.95,1.4) .. (10.93,3.29)   ;
%Straight Lines [id:da2930963755757128] 
\draw    (280.4,260.32) -- (279.81,42.93) ;
\draw [shift={(279.8,40.93)}, rotate = 449.84] [color={rgb, 255:red, 0; green, 0; blue, 0 }  ][line width=0.75]    (10.93,-3.29) .. controls (6.95,-1.4) and (3.31,-0.3) .. (0,0) .. controls (3.31,0.3) and (6.95,1.4) .. (10.93,3.29)   ;
%Shape: Ellipse [id:dp9939234516102196] 
\draw   (127.72,114.39) .. controls (127.72,94.37) and (195.95,78.15) .. (280.1,78.15) .. controls (364.25,78.15) and (432.47,94.37) .. (432.47,114.39) .. controls (432.47,134.4) and (364.25,150.63) .. (280.1,150.63) .. controls (195.95,150.63) and (127.72,134.4) .. (127.72,114.39) -- cycle ;
%Straight Lines [id:da4176606748779186] 
\draw    (432.47,114.39) -- (432.8,261.13) ;
%Straight Lines [id:da8871219632029788] 
\draw    (127.72,120.14) -- (127.8,261.13) ;
%Curve Lines [id:da5101437746675828] 
\draw    (127.8,261.13) .. controls (138.8,316.13) and (414.8,308.13) .. (432.8,261.13) ;
%Curve Lines [id:da7042964252877397] 
\draw  [dash pattern={on 4.5pt off 4.5pt}]  (127.8,261.13) .. controls (137.8,211.13) and (426.8,210.13) .. (432.8,261.13) ;
%Straight Lines [id:da3851586693872422] 
\draw  [dash pattern={on 4.5pt off 4.5pt}]  (280.4,260.32) -- (360.8,292.13) ;
%Straight Lines [id:da7299978368255808] 
\draw  [dash pattern={on 4.5pt off 4.5pt}]  (360.8,292.13) -- (360.8,191.13) ;
%Straight Lines [id:da989451393349547] 
\draw  [dash pattern={on 4.5pt off 4.5pt}]  (280.8,157.13) -- (326.28,174.2) -- (360.8,191.13) ;
%Curve Lines [id:da9702556493545555] 
\draw    (270.21,266.17) .. controls (278,269.6) and (287,269.24) .. (293,265.4) ;
%Straight Lines [id:da4395531121317289] 
\draw  [dash pattern={on 4.5pt off 4.5pt}]  (280.4,260.32) -- (360.8,191.13) ;
%Straight Lines [id:da07699363511693291] 
\draw    (360.8,191.13) -- (390.99,205.28) ;
\draw [shift={(392.8,206.13)}, rotate = 205.11] [color={rgb, 255:red, 0; green, 0; blue, 0 }  ][line width=0.75]    (10.93,-3.29) .. controls (6.95,-1.4) and (3.31,-0.3) .. (0,0) .. controls (3.31,0.3) and (6.95,1.4) .. (10.93,3.29)   ;
%Straight Lines [id:da1993839323449329] 
\draw    (360.8,191.13) -- (391.84,184.55) ;
\draw [shift={(393.8,184.13)}, rotate = 528.02] [color={rgb, 255:red, 0; green, 0; blue, 0 }  ][line width=0.75]    (10.93,-3.29) .. controls (6.95,-1.4) and (3.31,-0.3) .. (0,0) .. controls (3.31,0.3) and (6.95,1.4) .. (10.93,3.29)   ;
%Straight Lines [id:da28624261376962945] 
\draw    (360.8,191.13) -- (360.8,163.13) ;
\draw [shift={(360.8,161.13)}, rotate = 450] [color={rgb, 255:red, 0; green, 0; blue, 0 }  ][line width=0.75]    (10.93,-3.29) .. controls (6.95,-1.4) and (3.31,-0.3) .. (0,0) .. controls (3.31,0.3) and (6.95,1.4) .. (10.93,3.29)   ;
%Straight Lines [id:da555968168404436] 
\draw    (280.4,260.32) -- (296.8,260.27) -- (322,260.21) ;
\draw [shift={(324,260.2)}, rotate = 539.8399999999999] [color={rgb, 255:red, 0; green, 0; blue, 0 }  ][line width=0.75]    (10.93,-3.29) .. controls (6.95,-1.4) and (3.31,-0.3) .. (0,0) .. controls (3.31,0.3) and (6.95,1.4) .. (10.93,3.29)   ;
%Straight Lines [id:da7164434870481804] 
\draw    (280.4,260.32) -- (243.73,281.53) ;
\draw [shift={(242,282.53)}, rotate = 329.95] [color={rgb, 255:red, 0; green, 0; blue, 0 }  ][line width=0.75]    (10.93,-3.29) .. controls (6.95,-1.4) and (3.31,-0.3) .. (0,0) .. controls (3.31,0.3) and (6.95,1.4) .. (10.93,3.29)   ;
%Straight Lines [id:da4191268413254361] 
\draw    (280.4,260.32) -- (280.02,217.2) ;
\draw [shift={(280,215.2)}, rotate = 449.49] [color={rgb, 255:red, 0; green, 0; blue, 0 }  ][line width=0.75]    (10.93,-3.29) .. controls (6.95,-1.4) and (3.31,-0.3) .. (0,0) .. controls (3.31,0.3) and (6.95,1.4) .. (10.93,3.29)   ;

% Text Node
\draw (92.8,373.67) node [anchor=north west][inner sep=0.75pt]  [font=\small]  {$x$};
% Text Node
\draw (551,269.47) node [anchor=north west][inner sep=0.75pt]  [font=\small]  {$y$};
% Text Node
\draw (288,40.11) node [anchor=north west][inner sep=0.75pt]  [font=\small]  {$z$};
% Text Node
\draw (265,244.26) node [anchor=north west][inner sep=0.75pt]  [font=\footnotesize]  {$O$};
% Text Node
\draw (272.21,269.57) node [anchor=north west][inner sep=0.75pt]  [font=\small]  {$\phi $};
% Text Node
\draw (314,275) node [anchor=north west][inner sep=0.75pt]  [font=\small]  {$\rho $};
% Text Node
\draw (391.8,203.53) node [anchor=north west][inner sep=0.75pt]  [font=\small]  {$\overrightarrow{e_{\rho }}$};
% Text Node
\draw (394,160) node [anchor=north west][inner sep=0.75pt]  [font=\small]  {$\overrightarrow{e_{\phi }}$};
% Text Node
\draw (346,147) node [anchor=north west][inner sep=0.75pt]  [font=\small]  {$\overrightarrow{e_{z}}$};
% Text Node
\draw (266,163) node [anchor=north west][inner sep=0.75pt]  [font=\small]  {$z$};
% Text Node
\draw (333.74,188.33) node [anchor=north west][inner sep=0.75pt]  [font=\small]  {$P$};
% Text Node
\draw (234,261.27) node [anchor=north west][inner sep=0.75pt]  [font=\small]  {$\ot{i}$};
% Text Node
\draw (317.7,234.76) node [anchor=north west][inner sep=0.75pt]  [font=\small]  {$\ot{j}$};
% Text Node
\draw (266,202) node [anchor=north west][inner sep=0.75pt]  [font=\small]  {$\ot{k}$};


\end{tikzpicture}

    \end{center}
Mối liên hệ giữa toạ độ của một điểm trong hệ toạ độ Descartes và toạ độ của điểm đó trong hệ toạ độ trụ:
Trong toạ độ trụ, vector vị trí của một điểm có  thể được biểu diễn:
$$\ot{r}=\rho\cos{\theta}\ot{i}+\rho\sin{\theta}\ot{j}+z\ot{k},$$
Đạo hàm riêng vector $\ot{r}$ theo $\rho$, $\theta$, $z$ ta thu được:
\begin{align*}
    &\dfrac{\partial \ot{r}}{\partial \rho}= \cos{\theta}\ot{i}+\sin{\theta}\ot{j},\\
    &\dfrac{\partial \ot{r}}{\partial \theta}=-\rho\sin{\theta}\ot{i}+\rho\cos{\theta}\ot{j},\\
    &\dfrac{\partial \ot{r}}{\partial z}=\ot{k}.
\end{align*}
Các vector này có phương trùng với chiều tăng của $\rho$, $\theta$, $z$. Chia các vector này cho độ lớn của chính chúng ta thu được các vector đơn vị:
\begin{align*}
    &\ot{e_\rho}=\cos{\theta}\ot{i}+\sin{\theta}\ot{j},\\
    &\ot{e_\theta}=-\sin{\theta}\ot{i}+\cos{\theta}\ot{j},\\
    &\ot{e_z}=\ot{k}.
\end{align*}
Vi phân của vector vị trí $\ot{r}$ được biểu diễn:
\begin{align*}
    \dd\ot{r}&=\dfrac{\partial \ot{r}}{\partial \rho}\dd \rho+ \dfrac{\partial \ot{r}}{\partial \theta}\dd\theta+
    \dfrac{\partial \ot{r}}{\partial z}\dd z\\
    &=\left(\cos{\theta}\ot{i}+\sin{\theta}\ot{j}\right)\dd \rho+\rho\left(-\sin{\theta}\ot{i}+\cos{\theta}\ot{j}\right)\dd\theta+\ot{k}\dd z\\
    &=\dd \rho\ot{e_\rho}+\rho\dd\theta\ot{e_\theta}+\dd z\ot{e_z}.
\end{align*}
Vận tốc trong hệ toạ độ trụ:
\begin{align*}
    \dfrac{\dd \ot{r}}{\dd t}=\dfrac{\dd\rho}{\dd t}\ot{e_\rho}+\rho\dfrac{\dd\theta}{\dd t}\ot{e_\theta}+\dfrac{\dd z}{\dd t}\ot{e_z}.
    \end{align*}
Bằng cách đạo hàm vận tốc theo thời gian ta thu được biểu thức gia tốc của vật trong hệ toạ độ trụ:
\begin{align*}
    \dfrac{\dd^2 \ot{r}}{\dd t^2}=\left[\dfrac{\dd^2\rho}{\dd t^2}-\rho\left(\dfrac{\dd \theta}{\dd t}\right)^2\right]\ot{e_\rho}+\left(\rho\dfrac{\dd^2\theta}{\dd t^2}+2\dfrac{\dd \theta}{\dd t}\dfrac{\dd \rho}{\dd t}\right)\ot{e_\theta}+\left(\dfrac{\dd^2z}{\dd t^2}\right)\ot{e_z}.
\end{align*}
Các toán tử vector trong hệ toạ độ trụ:
\begin{align*}
    \ot{\nabla}\Phi&=\dfrac{\partial \Phi}{\partial \rho}\ot{e_\rho}+\dfrac{1}{\rho} \dfrac{\partial \Phi}{\partial \theta}\ot{e_\theta}+\dfrac{\partial \Phi}{\partial z}\ot{e_z},\\
    \ot{\nabla}\cdot\ot{a}&=\dfrac{1}{\rho}\dfrac{\partial}{\partial \rho}(\rho a_\rho)+\dfrac{1}{\rho}\dfrac{\partial a_\phi}{\partial \phi}+\dfrac{\partial a_z}{\partial z},\\
    \ot{\nabla}\times\ot{a}&=\left|\begin{array}{ccc}
    \ot{e_\rho} & \rho\ot{e_\theta} & \ot{e_z} \\
    \dfrac{\partial}{\partial\rho} & \dfrac{\partial}{\partial\theta} & \dfrac{\partial}{\partial z} \\
    a_\rho & a_\theta & a_z 
    \end{array}\right|,\\
    \ot{\nabla}^2\Phi&=\dfrac{1}{\rho}\dfrac{\partial}{\partial \rho}\left(\rho\dfrac{\partial\Phi}{ \partial\rho}\right)+\dfrac{1}{\rho^2}\dfrac{\partial^2\Phi}{ \partial \theta^2}+\dfrac{\partial^2\Phi}{\partial z^2}.
    \end{align*}
trong đó $\Phi$ là một trường vô hướng, còn $\ot{a}$ là một trường vector.
\subsubsection{Hệ toạ độ cầu.}
\begin{center}


\tikzset{every picture/.style={line width=0.75pt}} %set default line width to 0.75pt        

\begin{tikzpicture}[x=0.75pt,y=0.75pt,yscale=-1,xscale=1]
%uncomment if require: \path (0,438); %set diagram left start at 0, and has height of 438

%Straight Lines [id:da6020863454544023] 
\draw    (280.4,260.32) -- (548.8,261.13) ;
\draw [shift={(550.8,261.13)}, rotate = 180.17] [color={rgb, 255:red, 0; green, 0; blue, 0 }  ][line width=0.75]    (10.93,-3.29) .. controls (6.95,-1.4) and (3.31,-0.3) .. (0,0) .. controls (3.31,0.3) and (6.95,1.4) .. (10.93,3.29)   ;
%Straight Lines [id:da2807572666677913] 
\draw    (280.4,260.32) -- (270.21,266.17) -- (92.53,369.26) ;
\draw [shift={(90.8,370.27)}, rotate = 329.88] [color={rgb, 255:red, 0; green, 0; blue, 0 }  ][line width=0.75]    (10.93,-3.29) .. controls (6.95,-1.4) and (3.31,-0.3) .. (0,0) .. controls (3.31,0.3) and (6.95,1.4) .. (10.93,3.29)   ;
%Straight Lines [id:da2930963755757128] 
\draw    (280.4,260.32) -- (279.81,42.93) ;
\draw [shift={(279.8,40.93)}, rotate = 449.84] [color={rgb, 255:red, 0; green, 0; blue, 0 }  ][line width=0.75]    (10.93,-3.29) .. controls (6.95,-1.4) and (3.31,-0.3) .. (0,0) .. controls (3.31,0.3) and (6.95,1.4) .. (10.93,3.29)   ;
%Straight Lines [id:da555968168404436] 
\draw    (280.4,260.32) -- (296.8,260.27) -- (322,260.21) ;
\draw [shift={(324,260.2)}, rotate = 539.8399999999999] [color={rgb, 255:red, 0; green, 0; blue, 0 }  ][line width=0.75]    (10.93,-3.29) .. controls (6.95,-1.4) and (3.31,-0.3) .. (0,0) .. controls (3.31,0.3) and (6.95,1.4) .. (10.93,3.29)   ;
%Straight Lines [id:da7164434870481804] 
\draw    (280.4,260.32) -- (243.73,281.53) ;
\draw [shift={(242,282.53)}, rotate = 329.95] [color={rgb, 255:red, 0; green, 0; blue, 0 }  ][line width=0.75]    (10.93,-3.29) .. controls (6.95,-1.4) and (3.31,-0.3) .. (0,0) .. controls (3.31,0.3) and (6.95,1.4) .. (10.93,3.29)   ;
%Straight Lines [id:da4191268413254361] 
\draw    (280.4,260.32) -- (280.02,217.2) ;
\draw [shift={(280,215.2)}, rotate = 449.49] [color={rgb, 255:red, 0; green, 0; blue, 0 }  ][line width=0.75]    (10.93,-3.29) .. controls (6.95,-1.4) and (3.31,-0.3) .. (0,0) .. controls (3.31,0.3) and (6.95,1.4) .. (10.93,3.29)   ;
%Shape: Circle [id:dp7370290930091803] 
\draw   (128.54,260.32) .. controls (128.54,176.45) and (196.53,108.46) .. (280.4,108.46) .. controls (364.27,108.46) and (432.26,176.45) .. (432.26,260.32) .. controls (432.26,344.19) and (364.27,412.18) .. (280.4,412.18) .. controls (196.53,412.18) and (128.54,344.19) .. (128.54,260.32) -- cycle ;
%Curve Lines [id:da028684579275188415] 
\draw    (280.4,108.46) .. controls (411.8,149.13) and (411.8,390.13) .. (280.4,412.18) ;
%Curve Lines [id:da5558597959390716] 
\draw    (128.54,260.32) .. controls (167.8,346.13) and (398.8,345.13) .. (432.26,260.32) ;
%Straight Lines [id:da1319138831263973] 
\draw  [dash pattern={on 4.5pt off 4.5pt}]  (280.4,260.32) -- (280.4,412.18) ;
%Straight Lines [id:da9038005072678204] 
\draw  [dash pattern={on 4.5pt off 4.5pt}]  (280.4,260.32) -- (358.8,179.13) ;
%Straight Lines [id:da9002526366073975] 
\draw  [dash pattern={on 4.5pt off 4.5pt}]  (280.4,260.32) -- (373.8,308.13) ;
%Curve Lines [id:da48878651744379975] 
\draw    (280.2,237.76) .. controls (287,233.6) and (295,236.6) .. (298,241.6) ;
%Curve Lines [id:da9148348127380443] 
\draw    (261.2,271.43) .. controls (274,276.6) and (288,276.6) .. (301,270.6) ;
%Straight Lines [id:da7818762941417137] 
\draw    (358.8,179.13) -- (383.52,156.55) ;
\draw [shift={(385,155.2)}, rotate = 497.59] [color={rgb, 255:red, 0; green, 0; blue, 0 }  ][line width=0.75]    (10.93,-3.29) .. controls (6.95,-1.4) and (3.31,-0.3) .. (0,0) .. controls (3.31,0.3) and (6.95,1.4) .. (10.93,3.29)   ;
%Straight Lines [id:da8920292421548293] 
\draw    (358.8,179.13) -- (377.83,205.58) ;
\draw [shift={(379,207.2)}, rotate = 234.26] [color={rgb, 255:red, 0; green, 0; blue, 0 }  ][line width=0.75]    (10.93,-3.29) .. controls (6.95,-1.4) and (3.31,-0.3) .. (0,0) .. controls (3.31,0.3) and (6.95,1.4) .. (10.93,3.29)   ;
%Straight Lines [id:da6364354040222349] 
\draw    (358.8,179.13) -- (386.05,172.66) ;
\draw [shift={(388,172.2)}, rotate = 526.64] [color={rgb, 255:red, 0; green, 0; blue, 0 }  ][line width=0.75]    (10.93,-3.29) .. controls (6.95,-1.4) and (3.31,-0.3) .. (0,0) .. controls (3.31,0.3) and (6.95,1.4) .. (10.93,3.29)   ;

% Text Node
\draw (92.8,373.67) node [anchor=north west][inner sep=0.75pt]  [font=\small]  {$x$};
% Text Node
\draw (548,269.47) node [anchor=north west][inner sep=0.75pt]  [font=\small]  {$y$};
% Text Node
\draw (288,40.11) node [anchor=north west][inner sep=0.75pt]  [font=\small]  {$z$};
% Text Node
\draw (263,245.26) node [anchor=north west][inner sep=0.75pt]  [font=\small]  {$O$};
% Text Node
\draw (234,261.27) node [anchor=north west][inner sep=0.75pt]  [font=\small]  {$\ot{i}$};
% Text Node
\draw (317.7,234.76) node [anchor=north west][inner sep=0.75pt]  [font=\small]  {$\ot{j}$};
% Text Node
\draw (266,202) node [anchor=north west][inner sep=0.75pt]  [font=\small]  {$\ot{k}$};
% Text Node
\draw (289,222) node [anchor=north west][inner sep=0.75pt]  [font=\small]  {$\theta $};
% Text Node
\draw (281.2,274.83) node [anchor=north west][inner sep=0.75pt]  [font=\small]  {$\phi $};
% Text Node
\draw (329,211) node [anchor=north west][inner sep=0.75pt]  [font=\small]  {$\ot{r}$};
% Text Node
\draw (361,143.6) node [anchor=north west][inner sep=0.75pt]  [font=\small]  {$\overrightarrow{e_{r}}$};
% Text Node
\draw (385,172.6) node [anchor=north west][inner sep=0.75pt]  [font=\small]  {$\overrightarrow{e_{\phi }}$};
% Text Node
\draw (375,206.6) node [anchor=north west][inner sep=0.75pt]  [font=\small]  {$\overrightarrow{e_{\theta }}$};
% Text Node
\draw (338,170.6) node [anchor=north west][inner sep=0.75pt]  [font=\small]  {$P$};


\end{tikzpicture}

\end{center}
Mối liên hệ giữa toạ độ của một điểm trong hệ toạ độ Descartes và toạ độ của điểm đó trong hệ toạ độ cầu:
\begin{align*}
     x=r\sin{\theta}\cos{\phi},\text{   }y=r\sin{\theta}\sin{\phi},\text{   }z=r\cos{\theta}.
\end{align*}
Trong toạ độ cầu, vector vị trí của một điểm có  thể được biểu diễn:
$$\ot{r}=r\sin{\theta}\cos{\phi}\ot{i}+r\sin{\theta}\sin{\phi}\ot{j}+r\cos{\theta}\ot{k}.$$
Đạo hàm riêng vector $\ot{r}$ theo $r$, $\theta$, $\phi$ ta thu được:
\begin{align*}
    &\dfrac{\partial \ot{r}}{\partial r}= \sin{\theta}\cos{\phi}\ot{i}+\sin{\theta}\sin{\phi}\ot{j}+\cos{\theta}\ot{k},\\
    &\dfrac{\partial \ot{r}}{\partial \theta}=r\cos{\theta}\cos{\phi}\ot{i}+r\cos{\theta}\sin{\phi}\ot{j}-r\sin{\theta}\ot{k},\\
    &\dfrac{\partial \ot{r}}{\partial \phi}=-r\sin{\theta}\sin{\phi}\ot{i}+r\sin{\theta}\cos{\phi}\ot{j}.
\end{align*}
Chia các vector trên cho độ lớn của chính chúng ta thu được các vector đơn vị:
\begin{align*}
    &\ot{e_r}= \sin{\theta}\cos{\phi}\ot{i}+\sin{\theta}\sin{\phi}\ot{j}+\cos{\theta}\ot{k},\\
    &\ot{e_\theta}=\cos{\theta}\cos{\phi}\ot{i}+\cos{\theta}\sin{\phi}\ot{j}-\sin{\theta}\ot{k},\\
    &\ot{e_\phi}=\left(-\sin{\phi}\ot{i}+\cos{\phi}\ot{j}\right)\cdot\sin{\theta}.
\end{align*}
Vi phân toàn phần của $\ot{r}$:
$$\dd \ot{r}=\dd r\ot{e_r}+r\dd\theta\ot{e_\theta}+r\sin{\theta}\dd \phi\ot{e_\phi}.$$
Từ đó ta thu được biểu thức vận tốc và gia tốc trong hệ toạ độ cầu:
\begin{align*}
    \dfrac{\dd\ot{r}}{\dd t}&=\dfrac{\dd r}{\dd t}\ot{e_r}+r\dfrac{\dd \theta}{\dd t}\ot{e_\theta}+r\sin{\theta}\dfrac{\dd\phi}{\dd t}\ot{e_\phi},\\
    \dfrac{\dd^2\ot{r}}{\dd t^2}&=\left[\dfrac{\dd^2 r}{\dd t^2}-r\left(\dfrac{\dd \theta}{\dd t}\right)^2-r\left(\dfrac{\dd \phi}{\dd t}\right)^2\sin^2\theta \right] \ot{e_r}\\
    &+\left[r\dfrac{\dd^2 \theta}{\dd t^2}+2\dfrac{\dd r}{\dd t}\dfrac{\dd \theta}{\dd t}-r\left(\dfrac{\dd \phi}{\dd t}\right)^2\sin\theta\cos\theta\right]\ot{e_\theta}+\left[r\dfrac{\dd^2 \phi}{\dd t^2}\sin\theta+2\dfrac{\dd r}{\dd t}\dfrac{\dd \phi}{\dd t}\sin\theta+2\dfrac{\dd r}{\dd t}\dfrac{\dd \phi}{\dd t}\cos\theta\right]\ot{e_\phi}.
\end{align*}
Các toán tử vector trong hệ toạ độ cầu:
\begin{align*}
    \ot{\nabla}\Phi&=\dfrac{\partial \Phi}{\partial r}\ot{e_r}+\dfrac{1}{r} \dfrac{\partial \Phi}{\partial \theta}\ot{e_\theta}+\dfrac{1}{r\sin{\theta}}\dfrac{\partial \Phi}{\partial \theta}\ot{e_\theta},\\
    \ot{\nabla}\cdot\ot{a}&=\dfrac{1}{r^2}\dfrac{\partial}{\partial r}(r^2 a_r)+\dfrac{1}{r\sin{\theta}} \dfrac{\partial }{\partial \theta}(a_\theta\sin{\theta})+\dfrac{1}{r\sin{\theta}}\dfrac{\partial a_\phi}{\partial \phi},\\
    \ot{\nabla}\times\ot{a}&=\dfrac{1}{r^2\sin{\theta}}\left|\begin{array}{ccc}
    \ot{e_r} & r\ot{e_\theta} & r\sin{\theta}\ot{e_\phi} \\
    \dfrac{\partial}{\partial r} & \dfrac{\partial}{\partial\theta} & \dfrac{\partial}{\partial \phi} \\
    a_r & ra_\theta & r\sin{\theta}a_\phi
    \end{array}\right|,\\
    \ot{\nabla}^2\Phi&=\dfrac{1}{r^2}\dfrac{\partial}{ \partial r}\left(r^2\dfrac{\partial\Phi}{\partial r}\right)+\dfrac{1}{r^2\sin{\theta}}\dfrac{\partial}{\partial\theta}\left(\sin{\theta}\dfrac{\partial\Phi}{ \partial \theta}\right)+\dfrac{1}{r^2\sin^2{\theta}}\dfrac{\partial^2\Phi}{\partial \phi^2}.
    \end{align*}
trong đó $\Phi$ là một trường vô hướng và $\ot{a}$ là một trường vector.
\subsection{Tích phân.}
\subsubsection{Tích phân đường.}
Tích phân đường là phép tích phân một hàm số giữa hai điểm trong không gian dọc theo một đường cong $C$ cho trước. Ta có thể bắt gặp các dạng của tích phân đường:
$$\int_C \phi\dd \ot{r},\text{ }\int_C \ot{a}\cdot\dd \ot{r},\text{ }\int_C \ot{a}\times\dd \ot{r},$$
với $\phi$ là một trường vô hướng và $\ot{a}$ là một trường vector.\\
Dạng tích phân thứ hai là dạng được sử dụng phổ biến nhất khi ta giải các bài tập vật lý. Tích phân này có thể khai triển thành:
\begin{align*}
    \int_{C}\ot{a}\cdot\dd\ot{r}&=\int_C\left(a_x\ot{i}+a_y\ot{j}+a_z\ot{k}\right)\cdot\left(\dd x\ot{i}+\dd y\ot{j}+\dd z\ot{k}\right),\\
    &=\int_C\left(a_x\dd x+a_y\dd y+a_z\dd z\right),\\
    &=\int_Ca_x\dd x+\int_Ca_y\dd y+\int_Ca_z\dd z.
\end{align*}
\begin{center}


\tikzset{every picture/.style={line width=0.75pt}} %set default line width to 0.75pt        

\begin{tikzpicture}[x=0.7pt,y=0.7pt,yscale=-0.9,xscale=0.9]
%uncomment if require: \path (0,529); %set diagram left start at 0, and has height of 529

%Straight Lines [id:da44199661603776064] 
\draw    (291.61,248.48) -- (288.28,23.4) ;
\draw [shift={(288.25,21.4)}, rotate = 449.15] [color={rgb, 255:red, 0; green, 0; blue, 0 }  ][line width=0.75]    (10.93,-3.29) .. controls (6.95,-1.4) and (3.31,-0.3) .. (0,0) .. controls (3.31,0.3) and (6.95,1.4) .. (10.93,3.29)   ;
%Straight Lines [id:da35205079620063007] 
\draw    (291.61,248.48) -- (561,247.69) ;
\draw [shift={(563,247.68)}, rotate = 539.8299999999999] [color={rgb, 255:red, 0; green, 0; blue, 0 }  ][line width=0.75]    (10.93,-3.29) .. controls (6.95,-1.4) and (3.31,-0.3) .. (0,0) .. controls (3.31,0.3) and (6.95,1.4) .. (10.93,3.29)   ;
%Straight Lines [id:da1412527961654979] 
\draw    (291.61,248.48) -- (130.47,397.71) ;
\draw [shift={(129,399.07)}, rotate = 317.2] [color={rgb, 255:red, 0; green, 0; blue, 0 }  ][line width=0.75]    (10.93,-3.29) .. controls (6.95,-1.4) and (3.31,-0.3) .. (0,0) .. controls (3.31,0.3) and (6.95,1.4) .. (10.93,3.29)   ;
%Shape: Circle [id:dp4715715119517696] 
\draw  [fill={rgb, 255:red, 0; green, 0; blue, 0 }  ,fill opacity=1 ] (352,345.73) .. controls (352,345.18) and (351.55,344.73) .. (351,344.73) .. controls (350.45,344.73) and (350,345.18) .. (350,345.73) .. controls (350,346.29) and (350.45,346.73) .. (351,346.73) .. controls (351.55,346.73) and (352,346.29) .. (352,345.73) -- cycle ;
%Straight Lines [id:da35660399695696565] 
\draw    (291.61,248.48) -- (349.97,345.02) ;
\draw [shift={(351,346.73)}, rotate = 238.85] [color={rgb, 255:red, 0; green, 0; blue, 0 }  ][line width=0.75]    (10.93,-3.29) .. controls (6.95,-1.4) and (3.31,-0.3) .. (0,0) .. controls (3.31,0.3) and (6.95,1.4) .. (10.93,3.29)   ;
%Shape: Circle [id:dp39968124217131096] 
\draw  [fill={rgb, 255:red, 0; green, 0; blue, 0 }  ,fill opacity=1 ] (424,148.73) .. controls (424,148.18) and (423.55,147.73) .. (423,147.73) .. controls (422.45,147.73) and (422,148.18) .. (422,148.73) .. controls (422,149.29) and (422.45,149.73) .. (423,149.73) .. controls (423.55,149.73) and (424,149.29) .. (424,148.73) -- cycle ;
%Straight Lines [id:da492272563758009] 
\draw    (291.61,248.48) -- (422.4,149.94) ;
\draw [shift={(424,148.73)}, rotate = 503] [color={rgb, 255:red, 0; green, 0; blue, 0 }  ][line width=0.75]    (10.93,-3.29) .. controls (6.95,-1.4) and (3.31,-0.3) .. (0,0) .. controls (3.31,0.3) and (6.95,1.4) .. (10.93,3.29)   ;
%Curve Lines [id:da19527840465679325] 
\draw    (352,345.73) .. controls (379,333.73) and (342,305.73) .. (393,290.73) .. controls (444,275.73) and (360,238.73) .. (409,214.73) .. controls (458,190.73) and (437,173.07) .. (424,148.73) ;

% Text Node
\draw (331,353.4) node [anchor=north west][inner sep=0.75pt]  [font=\small]  {$A$};
% Text Node
\draw (417,123.4) node [anchor=north west][inner sep=0.75pt]  [font=\small]  {$B$};
% Text Node
\draw (267,229.4) node [anchor=north west][inner sep=0.75pt]  [font=\small]  {$O$};
% Text Node
\draw (264,22.4) node [anchor=north west][inner sep=0.75pt]  [font=\small]  {$z$};
% Text Node
\draw (118,372.4) node [anchor=north west][inner sep=0.75pt]  [font=\small]  {$x$};
% Text Node
\draw (558,219.4) node [anchor=north west][inner sep=0.75pt]  [font=\small]  {$y$};
% Text Node
\draw (400,224.13) node [anchor=north west][inner sep=0.75pt]  [font=\small]  {$C$};


\end{tikzpicture}

\end{center}
Xét một tích phân đường của vector trên một đường cong trơn $C$ kéo dài từ điểm $A$ tới điểm $B$. Thông thường, giá trị của tích phân đường sẽ phụ thuộc vào dạng đường đi, nhưng trong vài trường hợp đặc biệt, giá trị của tích phân này không phụ thuộc vào đường đi mà hoàn toàn phụ thuộc vào điểm đầu và điểm cuối.\\
Khi $\ot{a}=\ot{\nabla}\phi$, tích phân đường từ $A$ tới $B$:
\begin{align*}
    \int_A^B\ot{\nabla}\phi\cdot\dd\ot{r}=\int_a^B\left(\dfrac{\partial\phi}{\partial x}\dd x+\dfrac{\partial\phi}{\partial y}\dd y+\dfrac{\partial\phi}{\partial z}\dd z\right)
    =\int_A^B\dd \phi
    =\phi\left(\ot{r_a}\right)+\phi\left(\ot{r_B}\right).
\end{align*}
Tích phân đường của $\ot{\nabla}\phi$ trên một đường cong kín $C$ ($\ot{r_a}=\ot{r_B}$):
$$\oint_C\ot{\nabla}\phi\cdot\dd \ot{r}=0.$$
\subsubsection{Tích phân mặt.}
Tích phân mặt là tích phân của một hàm trên một bề mặt cho trước và có thể được coi như là tích phân kép của từng tích phân đường. Các loại tích phân mặt mà ta có thể bắt gặp:
$$\int_S \phi\dd \ot{S},\text{ }\int_S \ot{a}\cdot\dd \ot{S},\text{ }\int_S \ot{a}\times\dd \ot{S},$$
trong đó $\dd\ot{S}=\dd S\cdot \ot{n}$, với $\ot{n}$ là vector pháp tuyến đơn vị pháp tuyến của tiết diện $\dd S$.\\
Dạng thứ hai là dạng tích phân mặt được áp dụng phổ biến nhất trong vật lý:
$$\int_S \ot{a}\cdot\dd \ot{S}.$$
\subsubsection{Định lý Gauss.}
\begin{center}
    

\tikzset{every picture/.style={line width=0.75pt}} %set default line width to 0.75pt        

\begin{tikzpicture}[x=0.7pt,y=0.7pt,yscale=-0.9,xscale=0.9]
%uncomment if require: \path (0,529); %set diagram left start at 0, and has height of 529

%Straight Lines [id:da44199661603776064] 
\draw    (345.89,315.27) -- (342.69,81.33) ;
\draw [shift={(342.67,79.33)}, rotate = 449.22] [color={rgb, 255:red, 0; green, 0; blue, 0 }  ][line width=0.75]    (10.93,-3.29) .. controls (6.95,-1.4) and (3.31,-0.3) .. (0,0) .. controls (3.31,0.3) and (6.95,1.4) .. (10.93,3.29)   ;
%Straight Lines [id:da35205079620063007] 
\draw    (345.89,315.27) -- (578.07,315.31) ;
\draw [shift={(580.07,315.31)}, rotate = 180.01] [color={rgb, 255:red, 0; green, 0; blue, 0 }  ][line width=0.75]    (10.93,-3.29) .. controls (6.95,-1.4) and (3.31,-0.3) .. (0,0) .. controls (3.31,0.3) and (6.95,1.4) .. (10.93,3.29)   ;
%Straight Lines [id:da1412527961654979] 
\draw    (345.89,315.27) -- (191.41,470.32) ;
\draw [shift={(190,471.73)}, rotate = 314.9] [color={rgb, 255:red, 0; green, 0; blue, 0 }  ][line width=0.75]    (10.93,-3.29) .. controls (6.95,-1.4) and (3.31,-0.3) .. (0,0) .. controls (3.31,0.3) and (6.95,1.4) .. (10.93,3.29)   ;
%Straight Lines [id:da31541711755763036] 
\draw  [dash pattern={on 4.5pt off 4.5pt}]  (353.47,418.67) -- (378.24,393.84) ;
%Straight Lines [id:da04118861148400521] 
\draw  [dash pattern={on 4.5pt off 4.5pt}]  (378.24,393.84) -- (412.41,393.84) ;
%Straight Lines [id:da9593262298188945] 
\draw  [dash pattern={on 4.5pt off 4.5pt}]  (377.74,356.75) -- (378.24,393.84) ;
%Straight Lines [id:da5077214831967904] 
\draw    (411.92,356.75) -- (412.41,393.84) ;
%Straight Lines [id:da8420384200462303] 
\draw    (352.98,381.59) -- (353.47,418.67) ;
%Straight Lines [id:da5975780779194124] 
\draw    (387.15,381.59) -- (387.65,418.67) ;
%Straight Lines [id:da9036392756424103] 
\draw    (352.98,381.59) -- (377.74,356.75) ;
%Straight Lines [id:da2223524187032686] 
\draw    (387.65,418.67) -- (412.41,393.84) ;
%Straight Lines [id:da3868945523629139] 
\draw    (387.15,381.59) -- (411.92,356.75) ;
%Straight Lines [id:da006580156406035487] 
\draw    (377.74,356.75) -- (411.92,356.75) ;
%Straight Lines [id:da9523020467870091] 
\draw    (352.98,381.59) -- (387.15,381.59) ;
%Straight Lines [id:da14222214652908738] 
\draw    (353.47,418.67) -- (387.65,418.67) ;
%Straight Lines [id:da021561535908132257] 
\draw    (369.07,398.73) -- (339.48,428.32) ;
\draw [shift={(338.07,429.73)}, rotate = 315] [color={rgb, 255:red, 0; green, 0; blue, 0 }  ][line width=0.75]    (10.93,-3.29) .. controls (6.95,-1.4) and (3.31,-0.3) .. (0,0) .. controls (3.31,0.3) and (6.95,1.4) .. (10.93,3.29)   ;
%Straight Lines [id:da8869934352404589] 
\draw    (411.07,366.73) -- (426.69,350.19) ;
\draw [shift={(428.07,348.73)}, rotate = 493.36] [color={rgb, 255:red, 0; green, 0; blue, 0 }  ][line width=0.75]    (10.93,-3.29) .. controls (6.95,-1.4) and (3.31,-0.3) .. (0,0) .. controls (3.31,0.3) and (6.95,1.4) .. (10.93,3.29)   ;
%Straight Lines [id:da6111092919111751] 
\draw  [dash pattern={on 4.5pt off 4.5pt}]  (411.07,366.73) -- (397.07,381.73) ;

% Text Node
\draw (322.01,295.71) node [anchor=north west][inner sep=0.75pt]  [font=\small]  {$O$};
% Text Node
\draw (315,92.4) node [anchor=north west][inner sep=0.75pt]  [font=\small]  {$z$};
% Text Node
\draw (178,444.4) node [anchor=north west][inner sep=0.75pt]  [font=\small]  {$x$};
% Text Node
\draw (578,283.4) node [anchor=north west][inner sep=0.75pt]  [font=\small]  {$y$};
% Text Node
\draw (424,324.13) node [anchor=north west][inner sep=0.75pt]  [font=\small]  {$d\vec{S}$};
% Text Node
\draw (313,416.13) node [anchor=north west][inner sep=0.75pt]  [font=\small]  {$d\vec{S}$};
% Text Node
\draw (379,424.13) node [anchor=north west][inner sep=0.75pt]  [font=\small]  {$x+\Delta x$};
% Text Node
\draw (421,383.13) node [anchor=north west][inner sep=0.75pt]  [font=\small]  {$x$};


\end{tikzpicture}

\end{center}
Xét một phần tử thể tích $\Delta V=\Delta x\Delta y\Delta z$ được giới hạn từ $x$ tới $x+\Delta x$, từ $y$ tới $y+\Delta y$ và từ $z$ tới $x+\Delta z$. Tích phân của vector $\ot{a}$ trên toàn bộ mặt kín bao quanh $\Delta \tau$:
$$\oint_{\Delta S}\ot{a}\cdot\dd\ot{S}.$$
Trước tiên, ta xét trên hai mặt song song với mặt phẳng $y-z$, vị trí tại $x$ và $x+\Delta x$. Tại bề mặt ở $x$, $\dd \ot{S}=-\dd S\ot{i}$, tích phân trên mặt này:
\begin{align*}
    \int -a_x(x,y,z)\dd S \simeq  -a_x\left(x,y+\dfrac{\Delta y}{2},z+\dfrac{\Delta z}{2}\right)\Delta y\Delta z.
\end{align*}
Tại bề mặt ở $x+\Delta x$, $\dd \ot{S}=\dd S\ot{i}$, tích phân trên mặt này:
\begin{align*}
    \int a_x(x+\Delta x,y,z)\dd S \simeq  a_x\left(x+\Delta x,y+\dfrac{\Delta y}{2},z+\dfrac{\Delta z}{2}\right)\Delta y\Delta z.
\end{align*}
Khai triển hàm $a_x\left(x+\Delta x,y+\dfrac{\Delta y}{2},z+\dfrac{\Delta z}{2}\right)$ tới bậc nhất của $x$ ta thu được:
\begin{align*}
    a_x\left(x+\Delta x,y+\dfrac{\Delta y}{2},z+\dfrac{\Delta z}{2}\right)=a_x\left(x,y+\dfrac{\Delta y}{2},z+\dfrac{\Delta z}{2}\right)+\dfrac{\partial a_x}{\partial x}\Delta x.
\end{align*}
Tổng tích phân trên cả 2 bề mặt này:
\begin{align*}
    &\left[-a_x\left(x,y+\dfrac{\Delta y}{2},z+\dfrac{\Delta z}{2}\right)+a_x\left(x+\Delta x,y+\dfrac{\Delta y}{2},z+\dfrac{\Delta z}{2}\right)\right]\Delta y\Delta z\\
    =&\dfrac{\partial a_x}{\partial x}\Delta x\Delta y\Delta z
    =\dfrac{\partial a_x}{\partial x}\Delta V.
\end{align*}
Tương tự tích phân trên từng cặp bề mặt song song với $z-x$ và $x-y$:
$$\dfrac{\partial a_y}{\partial y}\Delta V,\text{ }\dfrac{\partial a_z}{\partial z}\Delta V.$$
Từ đó:
$$\oint_{\Delta S}\ot{a}\cdot\dd\ot{S}=\left(\dfrac{\partial a_x}{\partial x}+\dfrac{\partial a_y}{\partial y}+\dfrac{\partial a_z}{\partial z}\right)\Delta V=\left(\ot{\nabla}\cdot\ot{a}\right)\Delta V=\int_{\Delta V}\left(\ot{\nabla}\cdot\ot{a}\right)\dd V.$$
Xét thể tích $V$ được chia nhỏ thành nhiều phần, thể tích, tiết diện mỗi phần là $\Delta V_i$ và $\Delta S_i$. Tích phân mặt của vector $\ot{a}$ trên toàn bộ tiết diện $S$:
$$\oint_{S}\ot{a}\cdot\dd\ot{S}=\sum_i\oint_{\Delta S_i}\ot{a}\cdot\dd\ot{S}.$$
Tương tự với tích phân trên toàn bộ thể tích $V$:
$$\int_{V}\left(\ot{\nabla}\cdot\ot{a}\right)\dd V=\sum_i\int_{\Delta V_i}\left(\ot{\nabla}\cdot\ot{a}\right)\dd V.$$
Từ đó ta thu được biểu thức của định lý Gauss:
$$\oint_{S}\ot{a}\cdot\dd\ot{S}=\int_{V}\left(\ot{\nabla}\cdot\ot{a}\right)\dd V.$$
\subsubsection{Định lý Stokes.}
\begin{center}
	
	
	\tikzset{every picture/.style={line width=0.75pt}} %set default line width to 0.75pt        
	
	\begin{tikzpicture}[x=0.75pt,y=0.75pt,yscale=-1,xscale=1]
		%uncomment if require: \path (0,529); %set diagram left start at 0, and has height of 529
		
		%Straight Lines [id:da993211384778093] 
		\draw    (191,250.07) -- (190.01,13.73) ;
		\draw [shift={(190,11.73)}, rotate = 449.76] [color={rgb, 255:red, 0; green, 0; blue, 0 }  ][line width=0.75]    (10.93,-3.29) .. controls (6.95,-1.4) and (3.31,-0.3) .. (0,0) .. controls (3.31,0.3) and (6.95,1.4) .. (10.93,3.29)   ;
		%Straight Lines [id:da3076381842165865] 
		\draw    (191,250.07) -- (415,249.41) ;
		\draw [shift={(417,249.4)}, rotate = 539.8299999999999] [color={rgb, 255:red, 0; green, 0; blue, 0 }  ][line width=0.75]    (10.93,-3.29) .. controls (6.95,-1.4) and (3.31,-0.3) .. (0,0) .. controls (3.31,0.3) and (6.95,1.4) .. (10.93,3.29)   ;
		%Straight Lines [id:da26480340361088395] 
		\draw  [dash pattern={on 4.5pt off 4.5pt}]  (263,163.4) -- (263,249.73) ;
		%Straight Lines [id:da6840093811406123] 
		\draw  [dash pattern={on 4.5pt off 4.5pt}]  (333,163.4) -- (333,249.73) ;
		%Straight Lines [id:da9697912275902068] 
		\draw  [dash pattern={on 4.5pt off 4.5pt}]  (263,163.4) -- (190,163.73) ;
		%Straight Lines [id:da1226865054664732] 
		\draw  [dash pattern={on 4.5pt off 4.5pt}]  (263,123.4) -- (190,123.73) ;
		%Straight Lines [id:da0488554984631957] 
		\draw    (263,123.4) -- (333,123.4) ;
		\draw [shift={(298,123.4)}, rotate = 0] [fill={rgb, 255:red, 0; green, 0; blue, 0 }  ][line width=0.08]  [draw opacity=0] (8.93,-4.29) -- (0,0) -- (8.93,4.29) -- cycle    ;
		%Straight Lines [id:da8497518366955779] 
		\draw    (263,123.4) -- (263,163.4) ;
		\draw [shift={(263,143.4)}, rotate = 270] [fill={rgb, 255:red, 0; green, 0; blue, 0 }  ][line width=0.08]  [draw opacity=0] (8.93,-4.29) -- (0,0) -- (8.93,4.29) -- cycle    ;
		%Straight Lines [id:da26616847705821467] 
		\draw    (333,123.4) -- (333,163.4) ;
		\draw [shift={(333,143.4)}, rotate = 90] [fill={rgb, 255:red, 0; green, 0; blue, 0 }  ][line width=0.08]  [draw opacity=0] (8.93,-4.29) -- (0,0) -- (8.93,4.29) -- cycle    ;
		%Straight Lines [id:da9855840697667793] 
		\draw    (263,163.4) -- (333,163.4) ;
		\draw [shift={(298,163.4)}, rotate = 180] [fill={rgb, 255:red, 0; green, 0; blue, 0 }  ][line width=0.08]  [draw opacity=0] (8.93,-4.29) -- (0,0) -- (8.93,4.29) -- cycle    ;
		
		% Text Node
		\draw (412,266.4) node [anchor=north west][inner sep=0.75pt]  [font=\small]  {$y$};
		% Text Node
		\draw (316,261.4) node [anchor=north west][inner sep=0.75pt]  [font=\small]  {$y+\Delta y$};
		% Text Node
		\draw (254,257.4) node [anchor=north west][inner sep=0.75pt]  [font=\small]  {$y$};
		% Text Node
		\draw (144,114.4) node [anchor=north west][inner sep=0.75pt]  [font=\small]  {$z+\Delta z$};
		% Text Node
		\draw (164,159.4) node [anchor=north west][inner sep=0.75pt]  [font=\small]  {$z$};
		% Text Node
		\draw (168,11.4) node [anchor=north west][inner sep=0.75pt]  [font=\small]  {$z$};
		% Text Node
		\draw (169,251.4) node [anchor=north west][inner sep=0.75pt]  [font=\small]  {$O$};
		% Text Node
		\draw (253,102.4) node [anchor=north west][inner sep=0.75pt]  [font=\small]  {$A$};
		% Text Node
		\draw (246,173.4) node [anchor=north west][inner sep=0.75pt]  [font=\small]  {$B$};
		% Text Node
		\draw (346,168.4) node [anchor=north west][inner sep=0.75pt]  [font=\small]  {$C$};
		% Text Node
		\draw (340,110.4) node [anchor=north west][inner sep=0.75pt]  [font=\small]  {$D$};
		
		
	\end{tikzpicture}
	
\end{center}
Xét một phần tử diện tích $\Delta S= \Delta y \Delta z$ nằm trên mặt phẳng $Oyz$ được giới hạn từ $y$ tới $y+\Delta y$ và từ $z$ tới $z+\Delta z$. Tích phân đường của vector $\ot{a}$ trên toàn bộ đường giới hạn của thể tích $\Delta S$:
$$\oint_{\Delta l} \ot{a}\cdot\dd\ot{l}.$$
Trước hết, ta thử xem xét tích phân đường trên $2$ đoạn  $AB$ và $CD$
\begin{align*}
	  a_z\left(x+\Delta x,y+\Delta y,z+\dfrac{\Delta z}{2}\right)\Delta z-a_z\left(x+\Delta x,y,z+\dfrac{\Delta z}{2}\right)\Delta z\simeq \dfrac{\partial a_z}{\partial y}\Delta y\Delta z.
\end{align*}
Tương tự tích phân đường trên đoạn $BC$ và $DA$:
\begin{align*}
	a_y\left(x,y+\dfrac{\Delta y}{2},z\right)\Delta z-a_y\left(x,y+\dfrac{\Delta y}{2},z+\Delta z\right)\Delta z\simeq -\dfrac{\partial a_y}{\partial z}\Delta y\Delta z.
\end{align*}
Tích phân trên toàn bộ đường $ABCDA$:
\begin{align*}
	\oint_{\Delta l} \ot{a}\cdot\dd\ot{l}&=\left(\dfrac{\partial a_z}{\partial y}-\dfrac{\partial a_y}{\partial z}\right)\Delta y\Delta z=\left(\ot{\nabla}\times\ot{a}\right)_x\Delta S_x=\int_{\Delta S}\left(\ot{\nabla}\times\ot{a}\right)_x\dd S_x\\
	&=\int_{\Delta S}\left(\ot{\nabla}\times\ot{a}\right)\cdot\dd \ot{S}.
\end{align*}
Xét tiết diện $S$ được chia nhỏ thành nhiều phần, tiết diện, và chu vi đường giới hạn mỗi phần là $\Delta S_i$ và $\Delta l_i$. Tích phân mặt của vector $\ot{a}$ trên toàn bộ đường giới hạn của $S$:
$$\oint_{l}\ot{a}\cdot\dd\ot{l}=\sum_i\oint_{\Delta l_i}\ot{a}\cdot\dd\ot{l}.$$
Tương tự với tích phân trên toàn bộ tiết điện $S$:
$$\int_{S}\left(\ot{\nabla}\times\ot{a}\right)\cdot\dd \ot{S}=\sum_i\int_{\Delta S_i}\left(\ot{\nabla}\times\ot{a}\right)\cdot\dd \ot{S}.$$
Từ đó ta thu được biểu thức của định lý Stokes:
$$\oint_{l}\ot{a}\cdot\dd\ot{l}=\int_{S}\left(\ot{\nabla}\times\ot{a}\right)\cdot\dd \ot{S}.$$
\section{Lực tương tác tĩnh điện và điện trường.}
\subsection{Lực tương tác giữa hai điện tích.}
\begin{center}


\tikzset{every picture/.style={line width=0.75pt}} %set default line width to 0.75pt        

\begin{tikzpicture}[x=0.7pt,y=0.7pt,yscale=-0.9,xscale=0.9]
%uncomment if require: \path (0,529); %set diagram left start at 0, and has height of 529

%Straight Lines [id:da44199661603776064] 
\draw    (345.89,315.27) -- (344.02,118.4) ;
\draw [shift={(344,116.4)}, rotate = 449.46] [color={rgb, 255:red, 0; green, 0; blue, 0 }  ][line width=0.75]    (10.93,-3.29) .. controls (6.95,-1.4) and (3.31,-0.3) .. (0,0) .. controls (3.31,0.3) and (6.95,1.4) .. (10.93,3.29)   ;
%Straight Lines [id:da35205079620063007] 
\draw    (345.89,315.27) -- (578.07,315.31) ;
\draw [shift={(580.07,315.31)}, rotate = 180.01] [color={rgb, 255:red, 0; green, 0; blue, 0 }  ][line width=0.75]    (10.93,-3.29) .. controls (6.95,-1.4) and (3.31,-0.3) .. (0,0) .. controls (3.31,0.3) and (6.95,1.4) .. (10.93,3.29)   ;
%Straight Lines [id:da1412527961654979] 
\draw    (345.89,315.27) -- (206.4,456.64) ;
\draw [shift={(205,458.07)}, rotate = 314.62] [color={rgb, 255:red, 0; green, 0; blue, 0 }  ][line width=0.75]    (10.93,-3.29) .. controls (6.95,-1.4) and (3.31,-0.3) .. (0,0) .. controls (3.31,0.3) and (6.95,1.4) .. (10.93,3.29)   ;
%Straight Lines [id:da3762499529155341] 
\draw    (345.89,315.27) -- (374.42,408.82) ;
\draw [shift={(375,410.73)}, rotate = 253.04000000000002] [color={rgb, 255:red, 0; green, 0; blue, 0 }  ][line width=0.75]    (10.93,-3.29) .. controls (6.95,-1.4) and (3.31,-0.3) .. (0,0) .. controls (3.31,0.3) and (6.95,1.4) .. (10.93,3.29)   ;
%Straight Lines [id:da5272988446740561] 
\draw    (345.89,315.27) -- (496.25,232.7) ;
\draw [shift={(498,231.73)}, rotate = 511.22] [color={rgb, 255:red, 0; green, 0; blue, 0 }  ][line width=0.75]    (10.93,-3.29) .. controls (6.95,-1.4) and (3.31,-0.3) .. (0,0) .. controls (3.31,0.3) and (6.95,1.4) .. (10.93,3.29)   ;
%Straight Lines [id:da7738376186314102] 
\draw  [dash pattern={on 4.5pt off 4.5pt}]  (375,408.23) -- (498,231.73) ;
%Shape: Circle [id:dp7500394478396211] 
\draw  [color={rgb, 255:red, 0; green, 0; blue, 0 }  ,draw opacity=1 ][fill={rgb, 255:red, 0; green, 0; blue, 0 }  ,fill opacity=1 ] (372.5,408.23) .. controls (372.5,406.85) and (373.62,405.73) .. (375,405.73) .. controls (376.38,405.73) and (377.5,406.85) .. (377.5,408.23) .. controls (377.5,409.61) and (376.38,410.73) .. (375,410.73) .. controls (373.62,410.73) and (372.5,409.61) .. (372.5,408.23) -- cycle ;
%Shape: Circle [id:dp5204914324995789] 
\draw  [color={rgb, 255:red, 0; green, 0; blue, 0 }  ,draw opacity=1 ][fill={rgb, 255:red, 0; green, 0; blue, 0 }  ,fill opacity=1 ] (495.5,231.73) .. controls (495.5,230.35) and (496.62,229.23) .. (498,229.23) .. controls (499.38,229.23) and (500.5,230.35) .. (500.5,231.73) .. controls (500.5,233.11) and (499.38,234.23) .. (498,234.23) .. controls (496.62,234.23) and (495.5,233.11) .. (495.5,231.73) -- cycle ;
%Straight Lines [id:da9324145710247533] 
\draw    (498,231.73) -- (518.86,201.71) ;
\draw [shift={(520,200.07)}, rotate = 484.79] [color={rgb, 255:red, 0; green, 0; blue, 0 }  ][line width=0.75]    (10.93,-3.29) .. controls (6.95,-1.4) and (3.31,-0.3) .. (0,0) .. controls (3.31,0.3) and (6.95,1.4) .. (10.93,3.29)   ;

% Text Node
\draw (322.01,295.71) node [anchor=north west][inner sep=0.75pt]  [font=\small]  {$O$};
% Text Node
\draw (322,109.4) node [anchor=north west][inner sep=0.75pt]  [font=\small]  {$z$};
% Text Node
\draw (194,439.4) node [anchor=north west][inner sep=0.75pt]  [font=\small]  {$x$};
% Text Node
\draw (578,283.4) node [anchor=north west][inner sep=0.75pt]  [font=\small]  {$y$};
% Text Node
\draw (519,173.47) node [anchor=north west][inner sep=0.75pt]  [font=\small]  {$\ot{F}$};
% Text Node
\draw (376,420.47) node [anchor=north west][inner sep=0.75pt]  [font=\small]  {$q$};
% Text Node
\draw (502.5,235.13) node [anchor=north west][inner sep=0.75pt]    {$Q$};
% Text Node
\draw (340,356.47) node [anchor=north west][inner sep=0.75pt]  [font=\small]  {$\overrightarrow{r'}$};
% Text Node
\draw (421,245.6) node [anchor=north west][inner sep=0.75pt]  [font=\small]  {$\ot{r}$};


\end{tikzpicture}

\end{center}
    Điện tích điểm $q$ tác dụng lên điện tích điểm $Q$ (hai điện tích đặt trong chân không) một lực tương tác tĩnh điện:
    $$\ot{F}=\dfrac{1}{4\pi \varepsilon_0}\dfrac{qQ}{\left|\ot{r}-\ot{r'}\right|^3}\cdot\left(\ot{r}-\ot{r'}\right).$$
    với $\ot{r}$ là vector vị trí của $Q$ và $\ot{r'}$ là vector vị trí của $q$, 
%    $$k_e = \left\{ \begin{array}{cc}
%      \dfrac{1}{4\pi \varepsilon_0} = %8,987...\times10^9 N\cdot m^2 \cdot %C^{-2}\left(\text{hay } F^{-1}\cdot m\right)  & %\text{SI}, \\
 %    1 & \text{Gaussian.} \end{array}\right.$$
 $\varepsilon_0= 8,854... \times 10^{-12}~\mathrm{C^2\cdot N^{-1}\cdot m^{-2}}  (\text{hay}~\mathrm{F\cdot m^{-1}})$ là hằng số điện môi của chân không.
\subsection{Cường độ điện trường.}
\subsubsection{Cường độ điện trường gây bởi điện tích điểm.}
    Điện trường tạo ra tại điểm $S$ trong không gian (chân không hoặc không khí) gây bởi điện tích điểm $q$: 
    $$\ot{E}=\dfrac{1}{4\pi \varepsilon_0}\dfrac{q}{\left|\ot{r}-\ot{r'}\right|^3}\cdot\left(\ot{r}-\ot{r'}\right).$$
    trong đó $\ot{r}$ là vector vị trí của điểm $S$ và $\ot{r'}$ là vector vị trí của $q$.
    %hình1 
\subsubsection{Nguyên lý chồng chất điện trường.}
    Cường độ điện trường tại điểm $S$ gây bởi hệ điện tích $q_1,q_2,...,q_n$ bằng tổng cường độ điện trường của các điện tích riêng lẻ $q_1,q_2,...,q_n$ gây ra tại $S$:
    $$\ot{E}=\sum_{i=1}^{n}\ot{E}_i=\sum_{i=1}^{n}\dfrac{1}{4\pi \varepsilon_0}\dfrac{q_i}{\left|\ot{r}-\ot{r'_i}\right|^3}\cdot\left(\ot{r}-\ot{r'_i}\right).$$
    Điện trường gây ra bởi một phân bố điện tích bằng sự chồng chập của các thành phần nguyên tố $\dd q$:
    $$\ot{E}= \dfrac{1}{4\pi \varepsilon_0}\int\dfrac{\dd q}{\left|\ot{r}-\ot{r'}\right|^3}\cdot\left(\ot{r}-\ot{r'}\right).$$
    \begin{itemize}
    \item Đối với phân bố theo chiều dài:
    $$\ot{E}= \dfrac{1}{4\pi \varepsilon_0}\int\lambda\left(\ot{r'}\right)\dfrac{\ot{r}-\ot{r'}}{\left|\ot{r}-\ot{r'}\right|^3}\dd r'.$$
        \item Đối với phân bố theo bề mặt:
    $$\ot{E}=\dfrac{1}{4\pi \varepsilon_0}\int\sigma\left(\ot{r'}\right)\dfrac{\ot{r}-\ot{r'}}{\left|\ot{r}-\ot{r'}\right|^3}\dd S'.$$
    \item Đối với phân bố theo thể tích:
    $$\ot{E}=\dfrac{1}{4\pi \varepsilon_0}\int\varrho\left(\ot{r'}\right)\dfrac{\ot{r}-\ot{r'}}{\left|\ot{r}-\ot{r'}\right|^3}\dd V'.$$
    \end{itemize}
    
    
    Lực tĩnh điện tác dụng lên điện tích $q$ đặt trong điện trường $\ot{E}$:
    $$\ot{F}=q\ot{E}.$$
\subsection{Điện thông. Định luật Gauss $-$ Ostrogradsky.}
\subsubsection{Điện thông.}
\begin{center}
    

% Pattern Info
 
\tikzset{
pattern size/.store in=\mcSize, 
pattern size = 5pt,
pattern thickness/.store in=\mcThickness, 
pattern thickness = 0.3pt,
pattern radius/.store in=\mcRadius, 
pattern radius = 1pt}
\makeatletter
\pgfutil@ifundefined{pgf@pattern@name@_v65sgeu7h}{
\pgfdeclarepatternformonly[\mcThickness,\mcSize]{_v65sgeu7h}
{\pgfqpoint{0pt}{0pt}}
{\pgfpoint{\mcSize+\mcThickness}{\mcSize+\mcThickness}}
{\pgfpoint{\mcSize}{\mcSize}}
{
\pgfsetcolor{\tikz@pattern@color}
\pgfsetlinewidth{\mcThickness}
\pgfpathmoveto{\pgfqpoint{0pt}{0pt}}
\pgfpathlineto{\pgfpoint{\mcSize+\mcThickness}{\mcSize+\mcThickness}}
\pgfusepath{stroke}
}}
\makeatother
\tikzset{every picture/.style={line width=0.75pt}} %set default line width to 0.75pt        

\begin{tikzpicture}[x=0.75pt,y=0.75pt,yscale=-1,xscale=1]
%uncomment if require: \path (0,438); %set diagram left start at 0, and has height of 438

%Curve Lines [id:da02728278932272432] 
\draw    (173,123.93) .. controls (218,99.93) and (388,96.93) .. (426,118.93) ;
%Curve Lines [id:da10238125895061367] 
\draw    (158,228.93) .. controls (209,209.93) and (369,205.93) .. (411,223.93) ;
%Curve Lines [id:da13493424590328895] 
\draw    (173,123.93) .. controls (156,158.93) and (151,187.93) .. (158,228.93) ;
%Curve Lines [id:da4523259260078001] 
\draw    (426,118.93) .. controls (409,153.93) and (404,182.93) .. (411,223.93) ;
%Curve Lines [id:da4939641020183807] 
\draw    (87,102.93) .. controls (122,107.93) and (161,116.93) .. (195,164.93) ;
\draw [shift={(148.53,121.55)}, rotate = 28.03] [fill={rgb, 255:red, 0; green, 0; blue, 0 }  ][line width=0.08]  [draw opacity=0] (8.93,-4.29) -- (0,0) -- (8.93,4.29) -- cycle    ;
%Curve Lines [id:da28971858714178556] 
\draw    (174,55.8) .. controls (206,69.8) and (258,112.93) .. (283,153.93) ;
\draw [shift={(234.74,97.96)}, rotate = 41.6] [fill={rgb, 255:red, 0; green, 0; blue, 0 }  ][line width=0.08]  [draw opacity=0] (8.93,-4.29) -- (0,0) -- (8.93,4.29) -- cycle    ;
%Curve Lines [id:da670413941885676] 
\draw    (368,153) .. controls (365,112.93) and (348,73.8) .. (330,50.8) ;
\draw [shift={(356.04,99.24)}, rotate = 430.58000000000004] [fill={rgb, 255:red, 0; green, 0; blue, 0 }  ][line width=0.08]  [draw opacity=0] (8.93,-4.29) -- (0,0) -- (8.93,4.29) -- cycle    ;
%Curve Lines [id:da5923020412667761] 
\draw  [dash pattern={on 4.5pt off 4.5pt}]  (195,164.93) .. controls (210,188.93) and (220,213.93) .. (226,265.93) ;
%Curve Lines [id:da7093839559185253] 
\draw  [dash pattern={on 4.5pt off 4.5pt}]  (283,153.93) .. controls (298,191.8) and (304,223.8) .. (300,271.8) ;
%Curve Lines [id:da9833445858689851] 
\draw  [dash pattern={on 4.5pt off 4.5pt}]  (368,153) .. controls (368,190.93) and (370,223.93) .. (360,269.93) ;
%Straight Lines [id:da7765190544750014] 
\draw    (286.25,158.95) -- (256.98,106.55) ;
\draw [shift={(256,104.8)}, rotate = 420.81] [color={rgb, 255:red, 0; green, 0; blue, 0 }  ][line width=0.75]    (10.93,-3.29) .. controls (6.95,-1.4) and (3.31,-0.3) .. (0,0) .. controls (3.31,0.3) and (6.95,1.4) .. (10.93,3.29)   ;
%Shape: Parallelogram [id:dp8360893346521954] 
\draw  [pattern=_v65sgeu7h,pattern size=6pt,pattern thickness=0.75pt,pattern radius=0pt, pattern color={rgb, 255:red, 0; green, 0; blue, 0}] (278.75,149.93) -- (305,149.93) -- (293.75,167.97) -- (267.5,167.97) -- cycle ;
%Straight Lines [id:da4427953065536345] 
\draw    (286.25,158.95) -- (284.79,87.92) ;
\draw [shift={(284.75,85.92)}, rotate = 448.82] [color={rgb, 255:red, 0; green, 0; blue, 0 }  ][line width=0.75]    (10.93,-3.29) .. controls (6.95,-1.4) and (3.31,-0.3) .. (0,0) .. controls (3.31,0.3) and (6.95,1.4) .. (10.93,3.29)   ;
%Curve Lines [id:da29204072067186404] 
\draw    (274,137) .. controls (278,131.6) and (284,131.4) .. (286,135.6) ;

% Text Node
\draw (247,81.4) node [anchor=north west][inner sep=0.75pt]  [font=\small]  {$\ot{E}$};
% Text Node
\draw (277,61.4) node [anchor=north west][inner sep=0.75pt]  [font=\small]  {$\dd \ot{S}$};
% Text Node
\draw (271,116) node [anchor=north west][inner sep=0.75pt]  [font=\small]  {$\theta $};


\end{tikzpicture}

\end{center}
    Điện thông qua tiết diện $S$:
    %Hình 2
    $$\Phi=\int_S \ot{E}\cdot \dd \ot{S},$$
    với $\dd\ot{S}=\dd S\cdot \ot{n}$, trong đó $\ot{n}$ là vector pháp tuyến đơn vị pháp tuyến của $\dd S$.
    \subsubsection{Định luật Gauss $-$ Ostrogradsky.}
    Điện trường $\ot{E}$ tại một điểm trong không gian:
\begin{align*}
    \ot{E}&=\dfrac{1}{4\pi \varepsilon_0}\int\varrho\left(\ot{r'}\right)\dfrac{\ot{r}-\ot{r'}}{\left|\ot{r}-\ot{r'}\right|^3}\dd V',\\
    \Rightarrow \ot{\nabla}\cdot\ot{E}&=\dfrac{1}{4\pi \varepsilon_0}\int\ot{\nabla}\cdot\left(\dfrac{\ot{r}-\ot{r'}}{\left|\ot{r}-\ot{r'}\right|^3}\right)\varrho\left(\ot{r'}\right)\dd V',
\end{align*}
$\text{mà } \ot{\nabla}\cdot\left(\dfrac{\ot{r}-\ot{r'}}{\left|\ot{r}-\ot{r'}\right|^3}\right)=4\pi\delta^3\left(\ot{r}-\ot{r'}\right).$\\
Từ đó,\\
$$\ot{\nabla}\cdot\ot{E}=\dfrac{1}{ \varepsilon_0}\int\delta^3\left(\ot{r}-\ot{r'}\right)\varrho\left(\ot{r'}\right)\dd V',$$
Bởi vì $\int{\delta^3\left(\ot{r}-\ot{r'}\right)\varrho\left(\ot{r'}\right)\dd V'}=\varrho\left(\ot{r}\right)$, ta thu được biểu thức dạng vi phân của định luật Gauss:
$$\ot{\nabla}\cdot\ot{E}=\dfrac{\varrho\left(\ot{r}\right)}{ \varepsilon_0}.$$
\subsubsection{Định luật Gauss $-$ Ostrogradsky dạng tích phân.}
    \begin{align*}
      \left(\ot{\nabla}\cdot \ot{E}\right)&=\dfrac{\varrho}{ \varepsilon_0}\\
      \Rightarrow  \int_{V}\left(\ot{\nabla}\cdot \ot{E}\right)\dd V&=\int_{V}\dfrac{\varrho}{ \varepsilon_0}\dd V.
    \end{align*}
    Theo định lý phân kì:
    $$\int_{V}\left(\ot{\nabla}\cdot \ot{E}\right)\dd V=\oint_{S}\ot{E}\cdot \dd\ot{S}. $$
    Cuối cùng, ta thu được công thức:\\
    $$\Phi=\oint_{S}\ot{E}\cdot \dd\ot{S}=\int_{V}\dfrac{\varrho\dd V}{ \varepsilon_0} .$$
    Định luật Gauss có thể được phát biểu như sau: điện thông gửi qua một mặt kín $S$ (mặt Gauss) bằng tổng các điện tích trong mặt kín $S$ đó nhân với $4\pi k_e$.
\subsection{Lưu số của điện trường.}
    \begin{align*}
        \ot{E}&=\dfrac{1}{4\pi \varepsilon_0}\int\varrho\left(\ot{r'}\right)\dfrac{\ot{r}-\ot{r'}}{\left|\ot{r}-\ot{r'}\right|^3}\dd V'\\
        \Rightarrow \ot{\nabla}\times\ot{E}&=\dfrac{1}{4\pi \varepsilon_0}\int\ot{\nabla}\times\left(\dfrac{\ot{r}-\ot{r'}}{\left|\ot{r}-\ot{r'}\right|^3}\right)\varrho\left(\ot{r'}\right)\dd V',
    \end{align*}
    mà $\ot{\nabla}\times\left(\dfrac{\ot{r}-\ot{r'}}{\left|\ot{r}-\ot{r'}\right|^3}\right)=\ot{0}$ nên ta có:
    $$\ot{\nabla}\times\ot{E}=\ot{0}.$$
    Điện trường gây bởi điện tích điểm $q$ đặt tại gốc toạ độ:
    $$\ot{E}=\dfrac{1}{4\pi \varepsilon_0}\dfrac{q}{r^3}\ot{r}.$$
    Lưu số của điện trường trên một đường cong bất kì từ $A$ đến $B$ là:
    $$\int_{A}^{B}\ot{E}\cdot \dd\ot{r},$$
    trong toạ độ cầu, luôn có $\dd\ot{r}=\dd r\ot{e_r}+r\dd\theta\ot{e_\theta}+r\sin{\theta}\dd\varphi\ot{e_\varphi}$, từ đó:
    $$\int_{A}^{B}\ot{E}\cdot \dd\ot{l}=\int_{A}^{B}\dfrac{1}{4\pi \varepsilon_0}\dfrac{q}{r^2}\dd r.$$
    Lưu số của điện trường trên một đường cong kín bằng:
    \begin{align*}
    \oint\ot{E}\cdot \dd\ot{r}&=\oint \dfrac{1}{4\pi \varepsilon_0}\int\varrho\left(\ot{r'}\right)\dfrac{\ot{r}-\ot{r'}}{\left|\ot{r}-\ot{r'}\right|^3}\dd V'\cdot\dd \ot{r}\\
     &=\dfrac{1}{4\pi \varepsilon_0}\int\varrho\left(\ot{r'}\right)\dd V'\oint \dfrac{\ot{r}-\ot{r'}}{\left|\ot{r}-\ot{r'}\right|^3}\dd \ot{r}\\
    &=\dfrac{1}{4\pi \varepsilon_0}\int\varrho\left(\ot{r'}\right)\dd V'\oint \dfrac{\ot{r}-\ot{r'}}{\left|\ot{r}-\ot{r'}\right|^3}\dd \left(\ot{r}-\ot{r'}\right)\\
    &=\dfrac{1}{4\pi \varepsilon_0}\int\varrho\left(\ot{r'}\right)\dd V'\oint \dfrac{\dd\left|\ot{r}-\ot{r'}\right|}{\left|\ot{r}-\ot{r'}\right|^2}.
    \end{align*}
    Vì tích phân trên một mặt kín $\di\oint \dfrac{\dd\left|\ot{r}-\ot{r'}\right|}{\left|\ot{r}-\ot{r'}\right|^2}=0$ nên $\di\oint\ot{E}\cdot \dd\ot{r}=0$.
    Áp dụng định lý Stokes, ta thu được:
    $$\oint\ot{E}\cdot \dd\ot{r}=\int\left(\nabla \times \ot{E}\right)\cdot\dd \ot{S}=0.$$
\subsection{Điện thế, thế năng của điện tích trong điện trường.}
\subsubsection{Điện thế.}
    Điện thế là đại lượng đặc trưng của điện trường về khả năng thực hiện công và dự trữ thế năng.\\
    Phương trình Poisson:
    $$\nabla^2 \varphi+ \dfrac{\varrho}{ \varepsilon_0}=0.$$
    Mối liên hệ thế năng và điện trường:
    \begin{align*}
    \ot{E}&=-\ot{\nabla}\varphi, \\
    \text{hay } \dd \varphi&=-\ot{E}\cdot \dd \ot{r}. 
    \end{align*}
    Điện thế tại điểm có vị trí $r$ gây bởi điện trường của một điện tích điểm:
    $$\varphi(r)=-\int_{r_0}^{r}\ot{E}\cdot \dd\ot{r}= -\int_{r_0}^{r}\dfrac{1}{4\pi \varepsilon_0}\dfrac{Q}{r^3}\ot{r}\cdot  \dd\ot{r}=-\int_{r_0}^{r}\dfrac{1}{4\pi \varepsilon_0}\dfrac{Q}{r^2}\dd r=\dfrac{1}{4\pi \varepsilon_0}\dfrac{Q}{r}-\dfrac{1}{4\pi \varepsilon_0}\dfrac{Q}{r_0},$$
    với $\varphi(r_0)=\dfrac{1}{4\pi \varepsilon \varepsilon_0}\dfrac{Q}{r_0}$ là điện thế tại mốc $r_0$. Khi ta chọn một điểm ở xa vô cùng làm mốc:
    $$\varphi(r)=\dfrac{1}{4\pi \varepsilon_0}\dfrac{Q}{r}.$$
    Điện thế gây bởi hệ điện tích $Q_1,Q_2,...,Q_n$:
    $$\varphi=\sum_{i=1}^{n}\dfrac{1}{4\pi \varepsilon_0}\dfrac{Q_i}{r_i}.$$
    Điện thế gây bởi một phân bố điện tích:
    $$\varphi=\dfrac{1}{4\pi \varepsilon_0}\int \dfrac{\dd q}{\left|\ot{r}-\ot{r'}\right|}.$$
    \begin{itemize}
        \item Đối với phân bố theo chiều dài:
    $$\varphi=\dfrac{1}{4\pi \varepsilon_0}\int \dfrac{\lambda\left(\ot{r'}\right)}{\left|\ot{r}-\ot{r'}\right|}\dd\ot{l}'.$$
        \item Đối với phân bố theo bề mặt:
    $$\varphi=\dfrac{1}{4\pi \varepsilon_0}\int \dfrac{\sigma\left(\ot{r'}\right)}{\left|\ot{r}-\ot{r'}\right|}\dd\ot{S}'.$$
        \item Đối với phân bố theo thể tích:
    $$\varphi=\dfrac{1}{4\pi \varepsilon_0}\int \dfrac{\varrho\left(\ot{r'}\right)}{\left|\ot{r}-\ot{r'}\right|}\dd\ot{V}'.$$
    \end{itemize}
    
\subsubsection {Thế năng tĩnh điện.}
    Thế năng của điện tích điểm $q$ đặt trong điện trường:
    $$\dd U=-\ot{F}\cdot \dd\ot{r}=-q\ot{E}\cdot \dd\ot{r}=q\nabla\varphi\cdot \dd\ot{r}=q\cdot \dd\varphi.$$
    Tương tự điện trường, lực tác dụng lên một trường thế năng:
    $$\ot{F}=-\nabla U.$$ 
    Công cần thực hiện để làm dịch chuyển điện tích điểm $q$ từ vị trí $S_1$ tới vị trí $S_2$:
    $$A=U_2-U_1.$$
\subsubsection{Thế năng tương tác giữa các điện tích điểm.}
    Xét hai điện tích điểm $q_1$, $q_2$ ban đầu ở rất xa nhau, người ta phải thực hiện công để làm hai điện tích tới gần nhau. Công dịch chuyển từng điện tích:
    \begin{align*}
        \dd A_1=-\ot{{F}_{21}}\cdot \dd\ot{{r}_1},\\
        \dd A_2=-\ot{{F}_{12}}\cdot \dd\ot{{r}_2}.
    \end{align*}
    Công làm dịch chuyển cả hai điện tích:
    \begin{align*}
        &\dd A=\dd A_1+\dd A_2 \ \text{ mà } \  \ot{{F}_{21}}=-\ot{{F}_{12}}.\\
        \Rightarrow &\dd A=-\ot{{F}_{12}}\cdot \dd\ot{{r}_{12}}=-\dfrac{1}{4\pi \varepsilon_0}\dfrac{q_1q_2}{r_{12}^2}\dd r_{12}\\&\Rightarrow A=\dfrac{1}{4\pi \varepsilon_0}\dfrac{q_1q_2}{r_{12}}.
        \end{align*}
    Công cần thực hiện để dịch chuyển hai điện tích bằng độ chênh lệch thế năng        $A=V-V_\infty$, với $V_\infty$ là thế năng tương tác của hai điện tích khi chúng ở rất xa nhau và nếu chọn mốc thế năng khi khoảng cách hai điện tích là vô cùng thì $V_\infty=0$. Khi đó:\\
    $$V_{12}=V_{21}=\dfrac{1}{4\pi\varepsilon_0}\dfrac{q_1q_2}{r_{12}}.$$
    Thế năng tương tác của $1$ hệ các điện tích điểm $q_1,q_2,...,q_n$:\\
    $$U=\sum_{i =1}^{n}\sum_{j\ne i}^{n}\dfrac{1}{8\pi \varepsilon_0}\dfrac{q_iq_j}{r_{ij}},$$
    với $r_{ij}$ là khoảng các giữa hai điện tích $q_i$ và $q_j$.
    \begin{align*}
        U&=\dfrac{1}{2}\sum_{i =1}^{n}q_i\left(\sum_{j\ne i}^{n}\dfrac{1}{4\pi \varepsilon_0}\dfrac{q_j}{r_{ij}}\right)\\
        &=\dfrac{1}{2}\sum_{i =1}^{n}q_i\varphi\left(\ot{r_i}\right).
    \end{align*}
\subsubsection{Thế năng của một phân bố điện tích liên tục.}
    Thế năng của một phân bố điện tích:
    \begin{align*}
    W&=\dfrac{1}{2}\int\rho \varphi\dd V\\
    &=\dfrac{\varepsilon_0}{2}\int \left(\ot{\nabla}\cdot \ot{E}\right)\varphi\dd V 
    =\dfrac{\varepsilon_0}{2}\int \left(\ot{\nabla}\cdot \ot{E}\right)\varphi\dd V\\
    &=\dfrac{\varepsilon_0}{2}\int \left[\ot{\nabla}\cdot\left(\varphi\ot{E}\right)-\ot{E}\cdot\left(\ot{\nabla}\varphi\right)\right] \dd V\\
    &=\dfrac{\varepsilon_0}{2} \left[-\int\ot{E}\cdot\left(\ot{\nabla}\varphi\right)\dd V+\oint \varphi\ot{E}\cdot\dd \ot{S}\right] \\
    &=\dfrac{\varepsilon_0}{2} \left[\int E^2\dd V+\oint \varphi \ot{E}\cdot\dd \ot{S}\right].
    \end{align*}
    Ta biết rằng khi tính năng lượng tương tác của một phân bố điện tích, ta có thể lấy tích phân trên thể tích bất kì (miễn là thể tích đó chứa toàn bộ điện tích). Còn nếu lấy tích phân trên một bề mặt $S$ rất lớn thì tại đó $\ot{E}\rightarrow 0$ và tích phân $\di\oint V\ot{E}\cdot\dd \ot{S}$ tiến tới không. Vì vậy, thế năng của một phân bố điện tích có thể viết lại thành:
    $$W=\dfrac{\varepsilon_0}{2} \int E^2\dd V,$$
    tích phân này được lấy cho toàn bộ không gian.
\section{Lưỡng cực điện.}
\subsection{Moment lưỡng cực.}
    Moment lưỡng cực của hai điện tích điểm $q$ và $-q$:
    $$\ot{p}=q\cdot\ot{d}.$$
\subsection{Điện thế gây bởi lưỡng cực.}
     Điện thế tại một điểm $S$ trong không gian gây bởi điện trường của lưỡng cực điện:\\
     $$\varphi=\dfrac{1}{4\pi \varepsilon_0}\dfrac{q}{r_1}+\dfrac{1}{4\pi \varepsilon_0}\dfrac{-q}{r_2}.$$
    Sử dụng định lý hàm cos ta có thể tính được các khoảng $r_1$ và $r_2$:
    $$ r_1=\left(r^2+\dfrac{d^2}{4}-\dd r\cos{\theta}\right)^{\frac{1}{2}} \text{ và } r_2=\left(r^2+\dfrac{d^2}{4}+\dd r\cos{\theta}\right)^{\frac{1}{2}}.$$
    Thay vào biểu thức của điện thế đồng thời lấy gần đúng tới bậc nhất, ta thu được:
    $$\varphi=\dfrac{1}{4\pi \varepsilon_0 r}\left[\left(1+\dfrac{d}{2r}\cos{\theta}\right)-\left(1-\dfrac{d}{2r}\cos{\theta}\right)\right].$$
    Biểu thức điện thế gây bởi điện trường của điện tích:
    $$\varphi=\dfrac{1}{4\pi\varepsilon_0}\dfrac{qd\cos{\theta}}{r^2}=\dfrac{1}{4\pi \varepsilon_0}\dfrac{\ot{p}\cdot\ot{r}}{r^3}.$$
\subsection{Điện trường gây bởi lưỡng cực.}
    Ta có thể tính điện trường xung quanh lưỡng cực điện bằng cách lấy gradient của điện thế:
    $$\ot{E}=-\nabla\varphi.$$
    Trong toạ độ cầu ta thu được:
    \begin{align*}
    \ot{E}&=-\dfrac{\partial V}{\partial r}\ot{{e_r}}-\dfrac{1}{r}\dfrac{\partial V}{\partial \theta}\ot{{e}_\theta}-\dfrac{1}{r\sin{\theta}}\dfrac{\partial V}{\partial \varphi}\ot{{e}_\varphi}\\
    &=\dfrac{1}{4\pi \varepsilon_0}\dfrac{2p\cos{\theta}}{r^3}\hat{{e}}_r+ \dfrac{1}{4\pi \varepsilon_0}\dfrac{p\sin{\theta}}{r^3}\hat{{e}}_\theta\\
    &=\dfrac{1}{4\pi \varepsilon_0}\dfrac{2p\cos{\theta}\hat e_r+p\sin{\theta}\hat e_\theta}{r^3}.
    \end{align*}
    Từ đó, rút ra công thức tổng quát của điện trường gây bởi lưỡng cực điện:
    $$\ot{E}=\dfrac{1}{4\pi \varepsilon_0}\dfrac{3\left(\ot{p}\ot{{e_r}}\right)\ot{{e_r}}-\ot{p}}{r^3}.$$
\subsection{Tác dụng của điện trường lên lưỡng cực điện.}
\subsubsection{Điện trường đều.}
    Lưỡng cực đặt trong điện trường chịu tác dụng bởi một ngẫu lực. Điện trường đều tác dụng lên lưỡng cực điện một moment lưỡng cực:
    $$\ot{M}=\ot{p}\times\ot{E}.$$
\subsubsection{Điện trường không đều.}
    Tổng hợp lực tác dụng lên lưỡng cực khi nó được đặt trong một điện trường không đều:
    $$\ot{F}=q\ot{E}\left(\ot{r}+\dfrac{\ot{l}}{2}\right)-q\ot{E}\left(\ot{r}-\dfrac{\ot{l}}{2}\right),$$
    Lấy gần gần đúng tới bậc nhất:
    \begin{align*}
    E_x\left(\ot{r}+\dfrac{\ot{l}}{2}\right)&=E_x\left(\ot{r}\right)+\dfrac{\partial E_x}{\partial x}\dfrac{l_x}{2}+\dfrac{\partial E_x}{\partial y}\dfrac{l_y}{2}+\dfrac{\partial E_x}{\partial z}\dfrac{l_z}{2},\\
    &=E_x\left(\ot{r}\right)+\dfrac{\ot{l}}{2}\cdot \ot{\nabla} E_x.
    \end{align*}
    Lực tác dụng lên lưỡng cực theo phương $x$:
    $$F_x=q\ot{l}\cdot \ot{\nabla} E_x=\ot{p}\cdot\ot{\nabla} E_x=\left(\ot{p}\cdot\ot{\nabla}\right) E_x.$$
    Tương tự, ta cũng có thể xác định được lực điện trường theo phương $y$ và $z$, từ đó:
    $$\ot{F}=\left(\ot{p}\cdot\ot{\nabla}\right) \ot{E}.$$
\subsubsection{Thế năng tương tác giữa lưỡng cực và điện trường.}
    $$U=-\ot{p}\cdot\ot{E}.$$
\section{Điện trường trong vật dẫn và điện môi.}
\subsection{Cân bằng tĩnh điện của vật dẫn.}
Ở trạng thái cân bằng tĩnh điện,
\begin{itemize}
    \item Bên trong vật dẫn, các điện tích tự do không còn chuyển động có hướng nên không có lực điện trường. Chính vì thế, cường độ điện trường $
    \ot{{E}_\mathrm{int}}=\ot{0}$.
    \item Tương tự, ở mặt ngoài của vật dẫn, các điện tích tự do cũng không phải chịu tác dụng của lực theo phương tiếp tuyến. Vì vậy, điện trường luôn vuông góc với bề mặt của vật dẫn.
%hình 2
    \item Vật dẫn là vật đẳng thế tức là mọi điểm bên trong vật dẫn luôn có điện thế bằng nhau.
    \item Điện tích của vật dẫn chỉ phân bố trên bề mặt của vật dẫn và phân bố này phụ thuộc vào hình dạng của bề mặt vật dẫn. Đối với những vật có hình dạng cầu, mặt phẳng (hoặc trụ) vô hạn thì điện tích phân bố đều do tính chất đối xứng. Đối với vật dẫn có hình dạng bất kì, điện tích chủ yếu tập trung tại những vị trí bị lồi ra và thưa tại những chỗ bị lõm vào.
%hình3
\end{itemize} 
\subsection{Điện môi.}
\subsubsection{Sự phân cực điện môi.}
    Khi đặt một tấm điện môi vào một điện trường tạo bởi hai bản tụ song song, các điện tích âm bị hút về phía bản dương và các điện tích dương tới gần bản tụ âm làm xuất hiện các điện tích trên bề mặt thanh gọi là điện tích phân cực. Tuy nhiên bên trong điện môi gần như không tồn tại các điện tích tự do nên các điện tích phân cực khi xuất hiện sẽ cố định và không thể tách rời được. Vì vậy, chúng còn có tên gọi khác là điện tích liên kết.\\
    Độ phân cực điện:
    $$\ot{P}=
      \varepsilon_0\chi_e \ot{E},$$
    với $\ot{E}$ là điện trường bên trong điện môi, $\chi_e$ là độ cảm điện của điện môi và có quan hệ với hằng số điện môi:
    $$\varepsilon = 
    1+\chi_e.$$
    Mật độ mặt của điện tích liên kết xuất hiện trên tấm điện môi:\\
    $$\sigma=\ot{P}\cdot\hat{\ot{n}}.$$
    với $\hat{{n}}$ là vector pháp tuyến đơn vị tại bề mặt của điện môi.\\
    Mật độ khối của điện tích liên kết bên trong điện môi:
    $$\varrho_b=-\ot{\nabla}\cdot\ot{P}.$$
\subsubsection{Điện dịch.}
    Mật độ điện tích tổng cộng bên trong chất điện môi:
    $$\varrho=\varrho_f+\varrho_b,$$
    trong đó $\varrho_f$ là mật độ điện tích tự do và $\varrho_b$ là mật độ điện tích phân cực.\\
    Áp dụng định luật Gauss và công thức mật độ điện tích liên kết của chất điện môi:
    \begin{align*}
    &\varepsilon_0\ot{\nabla}\cdot\ot{E}
    =\varrho_f-\ot{\nabla}\cdot\ot{P},\\
    &\Rightarrow \ot{\nabla}\cdot\left(\varepsilon_0\ot{E}+\ot{P}\right)=\varrho_b,
    \end{align*}
    với $\ot{E}$ là điện trường tổng hợp bên trong điện môi được tạo bởi các điện tích tự do và bởi sự phân cực điện môi.\\
    Biểu thức trong ngoặc chính là vector điện dịch $\ot{D}$:
    \begin{align*}
    {D}&=\varepsilon_0\ot{E}+\ot{P}\\
     &= 
      \varepsilon_0\ot{E}+\ot{P}  \\
     &=
      \varepsilon\varepsilon_0\ot{E}.
    \end{align*}
     Đồng thời, ta cũng có công thức mối quan hệ giữa vector điện dịch $\ot{D}$ và mật độ điện tích tự do:
     \begin{align*}
        \ot{\nabla}\cdot\ot{D}&=\varrho_f,\\
        \text{và } \oint\ot{D}\cdot\dd \ot{S}&=\int\varrho_f\dd V.
     \end{align*}
\section{Dòng điện.}
\subsection{Phương trình liên tục.}
    Dòng điện qua một bề mặt $S$: 
    $$I=\int\ot{j}\cdot\dd \ot{S},$$
    với $\ot{j}$ là mật độ dòng điện trên một đơn vị diện tích.
    Dòng điện qua một tiết diện kín $S$:
    $$I=\oint_S\ot{j}\cdot\dd \ot{S}=\int_V\left(\ot{\nabla}\cdot\ot{j}\right)\dd V,$$
    mà dòng điện tổng qua mặt kín $S$ bằng độ thay đổi điện tích bên trong $S$ theo thời gian $I=-\dfrac{\dd}{\dd t}\int\varrho\dd V.$ Từ đó,\\
    \begin{align*}
            \int_V\left(\ot{\nabla}\cdot\ot{j}\right)\dd V=-\dfrac{\dd}{\dd t}\int\varrho\dd V=-\int\dfrac{\dd \varrho}{\dd t}\dd V,
    \end{align*}
    vì công thức trên đúng với mọi thể tích, nên ta thu được:
    $$\ot{\nabla}\cdot\ot{j}=-\dfrac{\dd \varrho}{\dd t}.$$
\subsection{Mô hình Drude.}
    Mô hình Drude về sự dẫn điện được đề xuất bởi Paul Drude vào năm $1900$ để mô tả các tính chất chuyển động của các electron trong vật liệu. Mô hình này cho rằng chuyển động của các electron có thể được biểu diễn như các hạt cổ điển và có thể sử dụng thuyết động học phân thử để mô tả tính chất của chuyển động này.\\
    Các giả thiết của mô hình Drude:
    \begin{itemize}
        \item Bỏ qua lực tương tác Coulomb giữa các electron hoặc electron với ion, electron chỉ tương tác thông qua việc va chạm với các ion cố định.
        \item Thời gian bay tự do trung bình $\tau$ của các electron và không phụ thuộc vào vị trí và vận tốc của chúng.
        \item Động lượng của electron bị mất ngay sau va chạm.
        \item Các electron đạt được cân bằng nhiệt động ngay qua việc va chạm với các ion.
    \end{itemize}
    Tại thời điểm $t$, electron có động lượng là $p(t)$. Sau một khoảng thời gian $\dd t$, động lượng của nó là $p(t+\dd t)$, trong khoảng thời gian này xác suất để nó thực hiện một va chạm là $\dfrac{\dd t}{\tau}$, những va chạm này làm giảm động lượng của electron đi một lượng là $\dfrac{\dd t}{\tau}p(t)$. 
    \begin{align*}
        &\ot{p}(t+\dd t)-\ot{p}(t)= -e\ot{E}\dd t-\dfrac{\dd t}{\tau}\ot{p}(t),\\
        \Rightarrow &\dd\ot{p}=-e\ot{E}\dd t-\dfrac{\dd t}{\tau}\ot{p},\\
        \Rightarrow &m_e\dfrac{\dd\ot{v_e}}{\dd t}=-e\ot{E}-\dfrac{m_e\ot{v_e}}{\tau}.
    \end{align*}
    Ở trạng thái mà electron chuyển động ổn định: 
    \begin{align*}
        &\dfrac{\dd\ot{v_e}}{\dd t}=\ot{0}\\
        &\ot{v_e}=\dfrac{-e\tau\ot{E}}{m_e}.
    \end{align*}
    Mật độ dòng điện qua một đơn vị diện tích:\\
    với $\sigma=e^2n_e\tau/m_e$ là điện dẫn suất và $\rho=1/\sigma$ là điện trở suất của vật liệu.\\
    Công suất toả nhiệt trên một đơn vị thể tích của vật liệu:
    $$w=\ot{J}\cdot\ot{E}=\sigma E^2=\dfrac{E^2}{\rho}.$$
\section{Tụ điện.}
    Tụ điện là một hệ gồm hai vật dẫn cách điện với nhau sao cho giữa chúng xảy ra hiện tượng hưởng ứng toàn phần.\\
    Điện dung của tụ điện:
    $$C=\dfrac{Q}{U},$$
    với $Q$ là điện tích trên bản tụ dương và $U$ là hiệu điện thế giữa hai bản tụ.
    \begin{itemize}
        \item Đối với tụ điện phẳng, hai bản tụ có tiết diện $S$ và cách nhau một khoảng là $d$:
    $$C=\dfrac{\varepsilon\varepsilon_0S}{d}.$$
        \item Đối với tụ điện cầu, bán kính hai bản tụ lần lượt là $R_1$và $R_2$:
    $$C=\dfrac{4\pi\varepsilon\varepsilon_0R_1R_2}{R_2-R_1}.$$
        \item Đối với tụ điện trụ, bán kính hai bản tụ là $R_1$ và $R_2$:
    $$C=\dfrac{2\pi\varepsilon\varepsilon_0l}{\ln\left(\dfrac{R_2}{R_1}\right)}.$$
    \end{itemize}
    
    
\section{Từ trường.}
%Phương trình Maxwell cho từ trường %$\ot{B}$:
%\begin{align*}
%    \nabla \cdot \ot{B}&=0,\\
%    \nabla\times\ot{B}&=4\pi k_m\ot{J}.
%\end{align*}
%    $\text{với }k_m = \left\{ %\begin{array}{cc}
%      \dfrac{\mu_0}{4\pi}= 10^{-7} T\cdot %m\cdot A^{-1} & \text{SI}, \\
%     1/c & \text{Gaussian.} %\end{array}\right.$\\
%     trong đó $\mu_0= 4\pi \cdot 10^{-7} %T\cdot m\cdot A^{-1}$ là hằng số từ, hay %được gọi là độ từ thẩm của chân không.
\subsection{Định luật Biot $-$ Savart.}
\begin{center}
\tikzset{every picture/.style={line width=0.75pt}} %set default line width to 0.75pt        

\begin{tikzpicture}[x=0.75pt,y=0.75pt,yscale=-1,xscale=1]
%uncomment if require: \path (0,529); %set diagram left start at 0, and has height of 529

%Straight Lines [id:da6198824201517081] 
\draw    (261,270.4) -- (261,82.4) ;
\draw [shift={(261,80.4)}, rotate = 450] [color={rgb, 255:red, 0; green, 0; blue, 0 }  ][line width=0.75]    (10.93,-3.29) .. controls (6.95,-1.4) and (3.31,-0.3) .. (0,0) .. controls (3.31,0.3) and (6.95,1.4) .. (10.93,3.29)   ;
%Straight Lines [id:da7332596736677781] 
\draw    (261,270.4) -- (497,271.39) ;
\draw [shift={(499,271.4)}, rotate = 180.24] [color={rgb, 255:red, 0; green, 0; blue, 0 }  ][line width=0.75]    (10.93,-3.29) .. controls (6.95,-1.4) and (3.31,-0.3) .. (0,0) .. controls (3.31,0.3) and (6.95,1.4) .. (10.93,3.29)   ;
%Straight Lines [id:da7660646480509843] 
\draw    (261,270.4) -- (100.69,372.33) ;
\draw [shift={(99,373.4)}, rotate = 327.55] [color={rgb, 255:red, 0; green, 0; blue, 0 }  ][line width=0.75]    (10.93,-3.29) .. controls (6.95,-1.4) and (3.31,-0.3) .. (0,0) .. controls (3.31,0.3) and (6.95,1.4) .. (10.93,3.29)   ;
%Curve Lines [id:da9029231400378166] 
\draw [color={rgb, 255:red, 155; green, 155; blue, 155 }  ,draw opacity=1 ]   (249,351.4) .. controls (274,349.4) and (302,362.4) .. (339,327.4) .. controls (376,292.4) and (398,337.4) .. (417,288.4) .. controls (436,239.4) and (469,248.9) .. (466,231.4) ;
\draw [shift={(297.62,350.59)}, rotate = 530.8299999999999] [fill={rgb, 255:red, 155; green, 155; blue, 155 }  ,fill opacity=1 ][line width=0.08]  [draw opacity=0] (6.25,-3) -- (0,0) -- (6.25,3) -- cycle    ;
\draw [shift={(382.48,313)}, rotate = 183.42] [fill={rgb, 255:red, 155; green, 155; blue, 155 }  ,fill opacity=1 ][line width=0.08]  [draw opacity=0] (6.25,-3) -- (0,0) -- (6.25,3) -- cycle    ;
\draw [shift={(438.59,255.93)}, rotate = 496.41] [fill={rgb, 255:red, 155; green, 155; blue, 155 }  ,fill opacity=1 ][line width=0.08]  [draw opacity=0] (6.25,-3) -- (0,0) -- (6.25,3) -- cycle    ;
%Straight Lines [id:da7678807329385631] 
\draw    (261,270.4) -- (337.39,326.22) ;
\draw [shift={(339,327.4)}, rotate = 216.16] [color={rgb, 255:red, 0; green, 0; blue, 0 }  ][line width=0.75]    (10.93,-3.29) .. controls (6.95,-1.4) and (3.31,-0.3) .. (0,0) .. controls (3.31,0.3) and (6.95,1.4) .. (10.93,3.29)   ;
%Straight Lines [id:da1198468311684675] 
\draw    (261,270.4) -- (390.52,152.74) ;
\draw [shift={(392,151.4)}, rotate = 497.75] [color={rgb, 255:red, 0; green, 0; blue, 0 }  ][line width=0.75]    (10.93,-3.29) .. controls (6.95,-1.4) and (3.31,-0.3) .. (0,0) .. controls (3.31,0.3) and (6.95,1.4) .. (10.93,3.29)   ;
%Straight Lines [id:da9166916629810289] 
\draw    (332,333.07) -- (350.4,319.27) ;
\draw [shift={(352,318.07)}, rotate = 503.13] [color={rgb, 255:red, 0; green, 0; blue, 0 }  ][line width=0.75]    (10.93,-3.29) .. controls (6.95,-1.4) and (3.31,-0.3) .. (0,0) .. controls (3.31,0.3) and (6.95,1.4) .. (10.93,3.29)   ;
%Shape: Circle [id:dp16969707352134855] 
\draw  [fill={rgb, 255:red, 0; green, 0; blue, 0 }  ,fill opacity=1 ] (332,332.07) .. controls (332,331.51) and (332.45,331.07) .. (333,331.07) .. controls (333.55,331.07) and (334,331.51) .. (334,332.07) .. controls (334,332.62) and (333.55,333.07) .. (333,333.07) .. controls (332.45,333.07) and (332,332.62) .. (332,332.07) -- cycle ;
%Shape: Circle [id:dp1966996744167231] 
\draw  [fill={rgb, 255:red, 0; green, 0; blue, 0 }  ,fill opacity=1 ] (350,318.07) .. controls (350,317.51) and (350.45,317.07) .. (351,317.07) .. controls (351.55,317.07) and (352,317.51) .. (352,318.07) .. controls (352,318.62) and (351.55,319.07) .. (351,319.07) .. controls (350.45,319.07) and (350,318.62) .. (350,318.07) -- cycle ;
%Straight Lines [id:da23448076425365016] 
\draw  [dash pattern={on 4.5pt off 4.5pt}]  (339,327.4) -- (392,151.4) ;
%Shape: Circle [id:dp26694221479053315] 
\draw  [fill={rgb, 255:red, 0; green, 0; blue, 0 }  ,fill opacity=1 ] (391,152.4) .. controls (391,151.85) and (391.45,151.4) .. (392,151.4) .. controls (392.55,151.4) and (393,151.85) .. (393,152.4) .. controls (393,152.95) and (392.55,153.4) .. (392,153.4) .. controls (391.45,153.4) and (391,152.95) .. (391,152.4) -- cycle ;
%Shape: Circle [id:dp7359717210428809] 
\draw  [fill={rgb, 255:red, 0; green, 0; blue, 0 }  ,fill opacity=1 ] (249,351.4) .. controls (249,350.85) and (249.45,350.4) .. (250,350.4) .. controls (250.55,350.4) and (251,350.85) .. (251,351.4) .. controls (251,351.95) and (250.55,352.4) .. (250,352.4) .. controls (249.45,352.4) and (249,351.95) .. (249,351.4) -- cycle ;
%Shape: Circle [id:dp6892229528825522] 
\draw  [fill={rgb, 255:red, 0; green, 0; blue, 0 }  ,fill opacity=1 ] (465,232.4) .. controls (465,231.85) and (465.45,231.4) .. (466,231.4) .. controls (466.55,231.4) and (467,231.85) .. (467,232.4) .. controls (467,232.95) and (466.55,233.4) .. (466,233.4) .. controls (465.45,233.4) and (465,232.95) .. (465,232.4) -- cycle ;

% Text Node
\draw (285,299.8) node [anchor=north west][inner sep=0.75pt]  [font=\small]  {$\ot{r}$};
% Text Node
\draw (312,187.47) node [anchor=north west][inner sep=0.75pt]  [font=\small]  {$\overrightarrow{r'}$};
% Text Node
\draw (344,328.97) node [anchor=north west][inner sep=0.75pt]  [font=\footnotesize]  {$\dd \ot{l}$};
% Text Node
\draw (429,238.8) node [anchor=north west][inner sep=0.75pt]  [font=\small]  {$I$};
% Text Node
\draw (241,82.8) node [anchor=north west][inner sep=0.75pt]  [font=\small]  {$x$};
% Text Node
\draw (79,362.8) node [anchor=north west][inner sep=0.75pt]  [font=\small]  {$y$};
% Text Node
\draw (498,281.8) node [anchor=north west][inner sep=0.75pt]  [font=\small]  {$z$};
% Text Node
\draw (239,251.8) node [anchor=north west][inner sep=0.75pt]  [font=\small]  {$O$};
% Text Node
\draw (381,134.8) node [anchor=north west][inner sep=0.75pt]  [font=\small]  {$S$};


\end{tikzpicture}

\end{center}
Cảm ứng từ tại điểm $S$ trong không gian gây bởi dòng điện ổn định:
    $$\ot{B}=\dfrac{\mu_0}{4\pi}\int\dfrac{I\dd \ot{ l'}\times\left(\ot{r}-\ot{r'}\right)}{\left|\ot{r}-\ot{r'}\right|^3}.$$
    với $\dd \ot{ l'}$ là một phần tử chiều dài của đòng điện có độ lớn $\dd l$ và có chiều trùng với chiều dòng điện.\\
    Cảm ứng từ gây bởi một phân bố dòng điện theo bề mặt:
    $$\ot{B}=\dfrac{\mu_0}{4\pi}\int\dfrac{\ot{j}\times\left(\ot{r}-\ot{r'}\right)}{\left|\ot{r}-\ot{r'}\right|^3}\dd V'.$$
    Áp dụng định luật Biot-Savart ta có thể tính được cảm ứng từ gây bởi một số dòng điện:
    \begin{itemize}
        \item Xét một dây dẫn mảnh có dạng đường tròn bán kính $R$, cảm ứng từ gây bởi dòng điện $I$ tại điểm $S$ nằm trên trục của đường tròn và cách tâm một đoạn $h$:
        $$B=2\pi k_m\dfrac{IR^2}{\left(R^2+h^2\right)}.$$
        \item Xét một đoạn dây dẫn thẳng, cảm ứng từ gây bởi đòng điện tại điểm $S$ cách dây dẫn một đoạn là $h$:
        \begin{align*}
        B=\dfrac{k_mI}{ h}\left(\cos{\theta_1}-\cos{\theta_2}\right).
        %HÌnh
        \end{align*}
        Đối với dây dẫn thẳng, dài vô hạn: $\theta_1\rightarrow 0,\text{ } \theta_2\rightarrow \pi$, khi đó:
        \[B=\dfrac{2 k_mI}{\pi h}.\]
        \item Xét một ống dây dẫn có dạng hình trụ bán kính $R$ có mật độ số vòng trên một đơn vị dài là $n$. Cảm ứng từ tại một điểm nằm trên trục của ống dây là:
        \begin{align*}
            B=2\pi k_mIn\left(\cos{\theta_2}-\cos{\theta_1}\right).
            %HÌNH
        \end{align*}
        Đối với ống dây dẫn dài vô hạn thì $\theta_1\rightarrow \pi,\text{ } \theta_2\rightarrow 0$, khi đó:
        \[ B=4\pi k_mIn.\]
    \end{itemize}
\subsection{Định lý Ampere.}
\subsubsection{Divergence của vector cảm ứng từ.}
Từ định luật Bio-Savart, công thức của cảm ứng từ $\ot{B}$:
$$\ot{B}=\dfrac{\mu_0}{4\pi}\int\dfrac{\ot{j}\left(\ot{r'}\right)\times\left(\ot{r}-\ot{r'}\right)}{\left|\ot{r}-\ot{r'}\right|^3}\dd V'.$$
Lấy div cả hai vế của phương trình:
\begin{align*}
    \ot{\nabla}\cdot\ot{B}=\dfrac{\mu_0}{4\pi}\int\ot{\nabla}\cdot\left[\dfrac{\ot{j}\times\left(\ot{r}-\ot{r'}\right)}{\left|\ot{r}-\ot{r'}\right|^3}\right]\dd V'.
\end{align*}
    Divergence của một tích hữu hướng:
    $$\ot{\nabla}\cdot\left[\ot{j}\times\dfrac{\left(\ot{r}-\ot{r'}\right)}{\left|\ot{r}-\ot{r'}\right|^3}\right]=\dfrac{\ot{r}-\ot{r'}}{\left|\ot{r}-\ot{r'}\right|^3}
    \cdot\left(\ot{\nabla}\times\ot{j}\right)-\ot{j}\cdot\left(\ot{\nabla}\times\dfrac{\ot{r}-\ot{r'}}{\left|\ot{r}-\ot{r'}\right|^3}\right).$$
    Vì $\ot{j}$ là hàm chỉ phụ thuộc vào vị trí của phần tử dòng điện $\ot{j}(x',y',z')$ nên đạo hàm của nó theo $x$, $y$ và $z$ bằng không. Trong khi đó, ta có thể dễ dàng chứng minh $$\ot{\nabla}\times\dfrac{\left(\ot{r}-\ot{r'}\right)}{\left|\ot{r}-\ot{r'}\right|^3}=0.$$
    Từ đó, ta thu được công thức:
    $$\ot{\nabla}\cdot\ot{B}=0.$$
\subsubsection{Rotation của vector cảm ứng từ.}
    $$\ot{\nabla}\times\ot{B}=\dfrac{\mu_0}{4\pi}\int\ot{\nabla}\times\left[\dfrac{\ot{j}\times\left(\ot{r}-\ot{r'}\right)}{\left|\ot{r}-\ot{r'}\right|^3}\right]\dd V'.$$
    Rot của một tích hữu hướng:
    \begin{align*}
    \ot{\nabla}\times\left[\dfrac{\ot{j}\times\left(\ot{r}-\ot{r'}\right)}{\left|\ot{r}-\ot{r'}\right|^3}\right]=\ot{j}\left(\ot{\nabla}\cdot\dfrac{\ot{r}-\ot{r'}}{\left|\ot{r}-\ot{r'}\right|^3}\right)-
    \left(\ot{j}\cdot\ot{\nabla}\right)\dfrac{\ot{r}-\ot{r'}}{\left|\ot{r}-\ot{r'}\right|^3},
    \end{align*}
    Vì $\ot{j}=\ot{j}\left(\ot{r'}\right)$ chỉ phụ thuộc vào $x'$, $y'$ và $z'$ nên ta bỏ qua các số hạng có đạo hàm của $\ot{j}$ theo $x$, $y$ và $z$ , còn $$\ot{\nabla}\cdot\left(\dfrac{\ot{r}-\ot{r'}}{\left|\ot{r}-\ot{r'}\right|^3}\right)=4\pi\delta^3\left(\ot{r}-\ot{r'}\right).$$
    Trong khi đó:
    \begin{align*}
    \left(\ot{j}\cdot\ot{\nabla}\right)\dfrac{\ot{r}-\ot{r'}}{\left|\ot{r}-\ot{r'}\right|^3}=-\left(\ot{j}\cdot\ot{\nabla}'\right)\dfrac{\ot{r}-\ot{r'}}{\left|\ot{r}-\ot{r'}\right|^3}.   
    \end{align*}
    Xét thành phần theo phương $x$:
    \begin{align*}
    \left(\ot{j}\cdot\ot{\nabla}'\right)\dfrac{x-x'}{\left|\ot{r}-\ot{r'}\right|^3} = \ot{\nabla}'\cdot\left(\dfrac{x-x'}{\left|\ot{r}-\ot{r'}\right|^3}\ot{j}\right)-\dfrac{x-x'}{\left|\ot{r}-\ot{r'}\right|^3}\left(\ot{\nabla}'\cdot\ot{j}\right).
    \end{align*}
    Khi dòng diện ổn định $\partial \varrho/\partial t=0$, áp dụng phương trình liên tục ta thu được $\ot{\nabla}'\cdot\ot{j}=0$.
    Còn số hạng thứ nhất, sau khi lấy tích phân trên toàn bộ thể tích chứa dòng điện ta được:
    \begin{align*}
        \int \ot{\nabla}'\cdot\left(\dfrac{x-x'}{\left|\ot{r}-\ot{r'}\right|^3}\ot{j}\right)\dd V'=\oint \dfrac{x-x'}{\left|\ot{r}-\ot{r'}\right|^3}\ot{j}\cdot\dd \ot{S'}.
    \end{align*}
    Nếu ta lấy tích phân trên một thể tích lớn hơn thể tích chứa dòng điện thì dòng điện tại bề mặt giới hạn của thể tích đó bằng không nên tích phân bề mặt ở vế phải của phương trình bằng không. Như vậy rot của $\ot{B}$ có thể được viết thành:
    \begin{align*}
     \ot{\nabla}\times\ot{B}&=\dfrac{\mu_0}{4\pi}\int\ot{j}\left(\ot{r'}\right)4\pi\delta^3\left(\ot{r}-\ot{r'}\right)\dd V,\\
     \Rightarrow\ot{\nabla}\times\ot{B} &=\mu_0\ot{j}\left(\ot{r}\right).
    \end{align*}
    \subsubsection{Lưu số của từ trường.}
    Trong từ trường, chọn một đường cong bất kì đi từ $A$ tới $B$. Lưu số của từ trường trên đường cong $AB$:
    $$C=\int_{A}^{B}\ot{B}\cdot \dd\ot{l}.$$ %HÌNH 
    Lưu số của từ trường trên đường cong kín :
    $$C=\oint \ot{B}\cdot \dd\ot{l}.$$
\subsubsection{Định luật Ampere.}
     Từ công thức div của $\ot{B}$: 
     \begin{align*}
        \ot{\nabla}\times\ot{B}&= \mu_0\ot{j},\\
        \Rightarrow \int \left(\ot{\nabla}\times\ot{B}\right)\cdot\dd \ot{S}&=\mu_0\int \ot{j}\cdot\dd\ot{S},
     \end{align*}
    Áp dụng định lý Stokes, ta thu được:
     $$\oint \ot{B}\cdot \dd\ot{l}=\mu_0\int \ot{j}\cdot\dd\ot{S}.$$
    Đây chính là công thức dạng tích phân của định luật Ampere.\\
    Ta có thể phát biểu định luật Ampere như sau: Lưu số của vector cường độ từ trường trên một đường cong kín bất kì bằng tổng đại số các dòng điện xuyên qua diện tích giới hạn bởi đường cong kín đó nhân với $\mu_0$.
    
     %HÌNH
\subsection{Lực từ.}
\subsubsection{Lực Lorentz.}
    Lực tác dụng lên điện tích $q$ đang chuyển động bên trong từ trường:
    $$\ot{F}=q\ot{v}\times\ot{B}.$$
\subsubsection{Lực từ tác dụng lên dòng điện.}
    Xét một dây dẫn đặt trong từ trường, lực từ tác dụng lên một phần tử $\dd\ot{l}$ của dây dẫn từng tổng hợp lực Lorentz tác dụng lên điện tích:
    \begin{align*}
    \dd\ot{F}=\sum Nq\ot{v}\times\ot{B}=(nS\dd l)q\ot{v}\times\ot{B}=(\ot{j}S\dd l)\times\ot{B},
    \end{align*}
    với $\ot{j}$ là vector mật độ dòng điện.\\
    Từ đó thu được công thức Ampere về lực do cảm ứng từ $\dd\ot{B}$ tác dụng lên phần tử dòng điện $I\dd\ot{l}$:
    $$\dd\ot{F}=I\dd\ot{l}\times\ot{B}.$$
\section{Hiệu ứng Hall.}
    \begin{center}
	
	
	\tikzset{every picture/.style={line width=0.75pt}} %set default line width to 0.75pt        
	
	\begin{tikzpicture}[x=0.75pt,y=0.75pt,yscale=-1,xscale=1]
		%uncomment if require: \path (0,529); %set diagram left start at 0, and has height of 529
		
		%Shape: Rectangle [id:dp1709980992121849] 
		\draw   (199,160.85) -- (402.84,160.85) -- (402.84,289) -- (199,289) -- cycle ;
		%Straight Lines [id:da17838120550375303] 
		\draw    (402.84,160.85) -- (457,57.73) ;
		%Straight Lines [id:da5245263651893479] 
		\draw    (199,160.85) -- (253.16,57.73) ;
		%Straight Lines [id:da7166032900763126] 
		\draw    (402.84,289) -- (457,185.88) ;
		%Straight Lines [id:da09105718305767985] 
		\draw  [dash pattern={on 4.5pt off 4.5pt}]  (199,289) -- (253.16,185.88) ;
		%Straight Lines [id:da422926910253522] 
		\draw    (253.16,57.73) -- (457,57.73) ;
		%Straight Lines [id:da17110647342597418] 
		\draw  [dash pattern={on 4.5pt off 4.5pt}]  (253.16,185.88) -- (457,185.88) ;
		%Straight Lines [id:da22178382158170806] 
		\draw    (457,57.73) -- (457,185.88) ;
		%Straight Lines [id:da8282885192830589] 
		\draw  [dash pattern={on 4.5pt off 4.5pt}]  (253.16,57.73) -- (253.16,185.88) ;
		%Shape: Ellipse [id:dp26473837766750274] 
		\draw   (329,201.37) .. controls (329,197.7) and (331.91,194.73) .. (335.5,194.73) .. controls (339.09,194.73) and (342,197.7) .. (342,201.37) .. controls (342,205.03) and (339.09,208) .. (335.5,208) .. controls (331.91,208) and (329,205.03) .. (329,201.37) -- cycle ;
		%Straight Lines [id:da8879282057890976] 
		\draw    (330.87,201.42) -- (340.12,201.32) ;
		%Straight Lines [id:da17681268957094032] 
		\draw    (329,201.37) -- (283.5,201.72) ;
		\draw [shift={(281.5,201.73)}, rotate = 359.56] [color={rgb, 255:red, 0; green, 0; blue, 0 }  ][line width=0.75]    (10.93,-3.29) .. controls (6.95,-1.4) and (3.31,-0.3) .. (0,0) .. controls (3.31,0.3) and (6.95,1.4) .. (10.93,3.29)   ;
		%Straight Lines [id:da9574139817588048] 
		\draw    (442,196) -- (423.06,226.37) ;
		\draw [shift={(422,228.07)}, rotate = 301.95] [color={rgb, 255:red, 0; green, 0; blue, 0 }  ][line width=0.75]    (10.93,-3.29) .. controls (6.95,-1.4) and (3.31,-0.3) .. (0,0) .. controls (3.31,0.3) and (6.95,1.4) .. (10.93,3.29)   ;
		%Straight Lines [id:da10859311646498071] 
		\draw    (335.5,194.73) -- (335.04,172.07) ;
		\draw [shift={(335,170.07)}, rotate = 448.84] [color={rgb, 255:red, 0; green, 0; blue, 0 }  ][line width=0.75]    (10.93,-3.29) .. controls (6.95,-1.4) and (3.31,-0.3) .. (0,0) .. controls (3.31,0.3) and (6.95,1.4) .. (10.93,3.29)   ;
		%Straight Lines [id:da152003714478401] 
		\draw    (335.5,208) -- (335.95,228.07) ;
		\draw [shift={(336,230.07)}, rotate = 268.7] [color={rgb, 255:red, 0; green, 0; blue, 0 }  ][line width=0.75]    (10.93,-3.29) .. controls (6.95,-1.4) and (3.31,-0.3) .. (0,0) .. controls (3.31,0.3) and (6.95,1.4) .. (10.93,3.29)   ;
		%Straight Lines [id:da4024817690374525] 
		\draw    (444,113) -- (444,156.73) ;
		\draw [shift={(444,158.73)}, rotate = 270] [color={rgb, 255:red, 0; green, 0; blue, 0 }  ][line width=0.75]    (10.93,-3.29) .. controls (6.95,-1.4) and (3.31,-0.3) .. (0,0) .. controls (3.31,0.3) and (6.95,1.4) .. (10.93,3.29)   ;
		%Straight Lines [id:da18595288142190936] 
		\draw    (354.87,261.42) -- (364.12,261.32) ;
		%Straight Lines [id:da4068985763354118] 
		\draw    (321.87,269.42) -- (331.12,269.32) ;
		%Straight Lines [id:da7456139955663048] 
		\draw    (278.87,267.42) -- (288.12,267.32) ;
		%Straight Lines [id:da20355007117178547] 
		\draw    (368.87,237.42) -- (378.12,237.32) ;
		%Straight Lines [id:da4997206575183537] 
		\draw    (235.87,260.42) -- (245.12,260.32) ;
		%Straight Lines [id:da5792639000107] 
		\draw    (294.98,103.07) -- (295.11,113.24) ;
		%Straight Lines [id:da9776533888337451] 
		\draw    (290.09,108.22) -- (300,108.09) ;
		%Straight Lines [id:da7137598301087202] 
		\draw    (343.98,85.07) -- (344.11,95.24) ;
		%Straight Lines [id:da9121157813027572] 
		\draw    (339.09,90.22) -- (349,90.09) ;
		%Straight Lines [id:da4716355113783621] 
		\draw    (344.98,109.07) -- (345.11,119.24) ;
		%Straight Lines [id:da4287654711539537] 
		\draw    (340.09,114.22) -- (350,114.09) ;
		%Straight Lines [id:da40212626161904486] 
		\draw    (298.98,133.07) -- (299.11,143.24) ;
		%Straight Lines [id:da0575957807820493] 
		\draw    (294.09,138.22) -- (304,138.09) ;
		%Straight Lines [id:da788861794658057] 
		\draw    (388.98,92.07) -- (389.11,102.24) ;
		%Straight Lines [id:da3207337424564627] 
		\draw    (384.09,97.22) -- (394,97.09) ;
		%Straight Lines [id:da7182144088121003] 
		\draw    (279.98,78.07) -- (280.11,88.24) ;
		%Straight Lines [id:da056725719084304016] 
		\draw    (275.09,83.22) -- (285,83.09) ;
		%Straight Lines [id:da47142030713382055] 
		\draw    (480.03,60) -- (481.97,184.4) ;
		\draw [shift={(482,186.4)}, rotate = 269.11] [color={rgb, 255:red, 0; green, 0; blue, 0 }  ][line width=0.75]    (10.93,-3.29) .. controls (6.95,-1.4) and (3.31,-0.3) .. (0,0) .. controls (3.31,0.3) and (6.95,1.4) .. (10.93,3.29)   ;
		\draw [shift={(480,58)}, rotate = 89.11] [color={rgb, 255:red, 0; green, 0; blue, 0 }  ][line width=0.75]    (10.93,-3.29) .. controls (6.95,-1.4) and (3.31,-0.3) .. (0,0) .. controls (3.31,0.3) and (6.95,1.4) .. (10.93,3.29)   ;
		%Straight Lines [id:da26851964312566334] 
		\draw    (503.16,189.82) -- (480.91,237.9) -- (457.84,287.59) ;
		\draw [shift={(457,289.4)}, rotate = 294.9] [color={rgb, 255:red, 0; green, 0; blue, 0 }  ][line width=0.75]    (10.93,-3.29) .. controls (6.95,-1.4) and (3.31,-0.3) .. (0,0) .. controls (3.31,0.3) and (6.95,1.4) .. (10.93,3.29)   ;
		\draw [shift={(504,188)}, rotate = 114.83] [color={rgb, 255:red, 0; green, 0; blue, 0 }  ][line width=0.75]    (10.93,-3.29) .. controls (6.95,-1.4) and (3.31,-0.3) .. (0,0) .. controls (3.31,0.3) and (6.95,1.4) .. (10.93,3.29)   ;
		
		% Text Node
		\draw (263,190.4) node [anchor=north west][inner sep=0.75pt]  [font=\small]  {$\vec{v}$};
		% Text Node
		\draw (338,162.4) node [anchor=north west][inner sep=0.75pt]  [font=\small]  {$-e\vec{E}$};
		% Text Node
		\draw (272,226.4) node [anchor=north west][inner sep=0.75pt]  [font=\small]  {$-e\vec{v} \times \overrightarrow{B_{0}}$};
		% Text Node
		\draw (405,197.4) node [anchor=north west][inner sep=0.75pt]  [font=\small]  {$\overrightarrow{B_{0}}$};
		% Text Node
		\draw (424,147.4) node [anchor=north west][inner sep=0.75pt]    {$\vec{E}$};
		% Text Node
		\draw (497,120.4) node [anchor=north west][inner sep=0.75pt]  [font=\small]  {$a$};
		% Text Node
		\draw (495,236.4) node [anchor=north west][inner sep=0.75pt]  [font=\small]  {$b$};
		
		
	\end{tikzpicture}
	
	\end{center}
    Hiệu ứng Hall là hiệu ứng xuất hiện các điện tích trái dấu trên bề mặt của một vật dẫn có dòng điện khi nó đặt trong từ trường. \\
    Xét một dây dẫn, tiết diện là hình chữ nhật với các cạnh $a$ và $b$, có dòng điện chạy qua. Dây dẫn này được đặt trong điện trường $\ot{E_0}$ và từ trường $\ot{B_0}$.\\
    Electron chuyển động trong dây đẫn chịu tác dụng của lực Lorentz, bị dịch chuyển và dần tập trung ở mặt trên, làm cho nó tích điện âm. Tương tự, bản bên dưới bị tích điện dương.\\
    %HÌNH
    Các điện tích tập trung ở 2 mặt lại tạo nên điện trường Hall. Lực mà điện trường này tác dụng lên electron:
    $$F_e=eE=e\cdot\dfrac{U}{a},$$
    Ở trạng thái ổn định, lực của điện trường Hall cân bằng với lực Lorentzz:
    $$e\cdot\dfrac{U}{a}=evB,$$
    Lại có mối quan hệ dòng điện $I$ và vận tốc chuyển động của electron:
    $$I=en_eabv,$$
    với $n_e$ là mật độ các electron tự do trong kim loại.\\
    Từ đó:
    $$U=\dfrac{IB}{en_eb}=R_H\cdot\dfrac{IB}{b},$$
    với $R_H=\dfrac{1}{en_e}$ là hằng số Hall của kim loại.
    
    \section{Lưỡng cực từ.}
    Moment lưỡng cực từ của một vòng dây có dòng điện I chạy qua:
    $$\ot{p_m}=I\ot{S}=IS\ot{n}.$$
    với $S$ là diện tích giới hạn bởi vòng dây còn $\ot{n}$ là vector pháp tuyến đơn vị của mặt phẳng dòng điện và chiều của nó có thể được xác định bằng quy tắc đinh ốc.\\
    Từ trường gây bởi lưỡng cực từ $\ot{p_m}$ tại một điểm rất xa trong không gian:
    $$\ot{B}=k_m\dfrac{3\left(\ot{p_m}\times\ot{e_r}\right)-\ot{p_m}}{r^3}.$$
\section{Từ trường trong vật chất.}
\subsection{Sự từ hoá của các chất.}
    Các chất khi đặt trong từ trường đều bị nhiễm từ (hay bị từ hoá). Khi đặt trong từ trường ngoài các moment lưỡng cực có xu hướng bị quay đi theo từ trường.\\
    Vector từ hoá của một chất là mật độ các moment từ trong một đơn vị thể tích của chất:
    $$\ot{M}=\dfrac{\dd\ot{p_m}}{\dd V}.$$
    Khi đặt trong từ trường càng mạnh thì vector từ hoá càng lớn:
    $$\ot{M}=\chi_m\dfrac{\ot{B}}{\mu_0}.$$
    trong đó $\ot{H}$ là vector cường độ từ trường bên trong vật, $\chi_m$ là độ cảm từ và có quan hệ với độ từ thẩm của vật:
   $$\mu = 
      1+\chi_m.$$
    Dựa vào tính chất và mức độ từ hoá, người ta có thể phân loại các chất:
    \begin{itemize}
        \item Chất thuận từ: $\chi_m<0 $ và $\mu>1$
        \item Chất nghịch từ: $\chi_m < 0$ và $\mu<1$
    \end{itemize}
    Đối với chất thuận từ và nghịch từ thì $|\chi_m|\ll1$ do đó $\mu\approx 1.$
     \begin{itemize}
        \item Chất sắt từ: $\chi_m\gg 1$ nên $\mu\gg 1.$
    \end{itemize}
    Sự quay của lưỡng cực do tác dụng của từ trường làm xuất hiện dòng điện bên trong chất bị từ hoá, dòng điện này được gọi là dòng liên kết:
    $$\ot{j_b}=\ot{\nabla}\times\ot{M}.$$
    với $\ot{j_b}$ là mật độ dòng điện liên kết.
\subsection{Cường độ từ trường.}
    Dòng điện trong vật chất là tổng hợp của dòng điện liên kết và dòng điện gây bởi các nguồn khác (gọi là dòng điện tự do):
    $$\ot{j}=\ot{j_b}+\ot{j_f}$$
    Áp dụng định luật Ampere và công thức dòng điện liên kết:
    $$\ot{\nabla}\times\left(\dfrac{1}{\mu_0}\ot{B}-\ot{M}\right)=\ot{j_f}.$$
    Đại lượng bên trong ngoặc được gọi là vector cường độ từ trường:
    $$\ot{H}=\dfrac{1}{\mu_0}\ot{B}-\ot{M}$$
    vector cường độ từ trường bên trong vật khi được đặt trong một từ trường ngoài $\ot{B}_{0}$:
    $$\ot{H} = 
      \dfrac{\ot{B}}{\mu_0}-\ot{M}.$$
     Từ đó, ta có thể suy ra được mối quan hệ của vector cường độ từ trường $\ot{H}$ và vector cảm ứng từ $\ot{B}$:
     $$\ot{B} = 
      \mu\mu_0\ot{H}.$$
    Đồng thời ta cũng thu được biểu thức của định luật Ampere bên trong vật chất:
    \begin{align*}
        \ot{\nabla}\times\ot{H}&=\ot{j_f},\\
        \text{và } \oint\ot{H}\times\dd \ot{l}&=\int\ot{j_f}\cdot\dd \ot{S}.
     \end{align*}
\section{Cảm ứng điện từ.}
\subsection{Từ thông.}
\begin{center}
    

% Pattern Info
 
\tikzset{
pattern size/.store in=\mcSize, 
pattern size = 5pt,
pattern thickness/.store in=\mcThickness, 
pattern thickness = 0.3pt,
pattern radius/.store in=\mcRadius, 
pattern radius = 1pt}
\makeatletter
\pgfutil@ifundefined{pgf@pattern@name@_v65sgeu7h}{
\pgfdeclarepatternformonly[\mcThickness,\mcSize]{_v65sgeu7h}
{\pgfqpoint{0pt}{0pt}}
{\pgfpoint{\mcSize+\mcThickness}{\mcSize+\mcThickness}}
{\pgfpoint{\mcSize}{\mcSize}}
{
\pgfsetcolor{\tikz@pattern@color}
\pgfsetlinewidth{\mcThickness}
\pgfpathmoveto{\pgfqpoint{0pt}{0pt}}
\pgfpathlineto{\pgfpoint{\mcSize+\mcThickness}{\mcSize+\mcThickness}}
\pgfusepath{stroke}
}}
\makeatother
\tikzset{every picture/.style={line width=0.75pt}} %set default line width to 0.75pt        

\begin{tikzpicture}[x=0.75pt,y=0.75pt,yscale=-1,xscale=1]
%uncomment if require: \path (0,438); %set diagram left start at 0, and has height of 438

%Curve Lines [id:da02728278932272432] 
\draw    (173,123.93) .. controls (218,99.93) and (388,96.93) .. (426,118.93) ;
%Curve Lines [id:da10238125895061367] 
\draw    (158,228.93) .. controls (209,209.93) and (369,205.93) .. (411,223.93) ;
%Curve Lines [id:da13493424590328895] 
\draw    (173,123.93) .. controls (156,158.93) and (151,187.93) .. (158,228.93) ;
%Curve Lines [id:da4523259260078001] 
\draw    (426,118.93) .. controls (409,153.93) and (404,182.93) .. (411,223.93) ;
%Curve Lines [id:da4939641020183807] 
\draw    (87,102.93) .. controls (122,107.93) and (161,116.93) .. (195,164.93) ;
\draw [shift={(148.53,121.55)}, rotate = 28.03] [fill={rgb, 255:red, 0; green, 0; blue, 0 }  ][line width=0.08]  [draw opacity=0] (8.93,-4.29) -- (0,0) -- (8.93,4.29) -- cycle    ;
%Curve Lines [id:da28971858714178556] 
\draw    (174,55.8) .. controls (206,69.8) and (258,112.93) .. (283,153.93) ;
\draw [shift={(234.74,97.96)}, rotate = 41.6] [fill={rgb, 255:red, 0; green, 0; blue, 0 }  ][line width=0.08]  [draw opacity=0] (8.93,-4.29) -- (0,0) -- (8.93,4.29) -- cycle    ;
%Curve Lines [id:da670413941885676] 
\draw    (368,153) .. controls (365,112.93) and (348,73.8) .. (330,50.8) ;
\draw [shift={(356.04,99.24)}, rotate = 430.58000000000004] [fill={rgb, 255:red, 0; green, 0; blue, 0 }  ][line width=0.08]  [draw opacity=0] (8.93,-4.29) -- (0,0) -- (8.93,4.29) -- cycle    ;
%Curve Lines [id:da5923020412667761] 
\draw  [dash pattern={on 4.5pt off 4.5pt}]  (195,164.93) .. controls (210,188.93) and (220,213.93) .. (226,265.93) ;
%Curve Lines [id:da7093839559185253] 
\draw  [dash pattern={on 4.5pt off 4.5pt}]  (283,153.93) .. controls (298,191.8) and (304,223.8) .. (300,271.8) ;
%Curve Lines [id:da9833445858689851] 
\draw  [dash pattern={on 4.5pt off 4.5pt}]  (368,153) .. controls (368,190.93) and (370,223.93) .. (360,269.93) ;
%Straight Lines [id:da7765190544750014] 
\draw    (286.25,158.95) -- (256.98,106.55) ;
\draw [shift={(256,104.8)}, rotate = 420.81] [color={rgb, 255:red, 0; green, 0; blue, 0 }  ][line width=0.75]    (10.93,-3.29) .. controls (6.95,-1.4) and (3.31,-0.3) .. (0,0) .. controls (3.31,0.3) and (6.95,1.4) .. (10.93,3.29)   ;
%Shape: Parallelogram [id:dp8360893346521954] 
\draw  [pattern=_v65sgeu7h,pattern size=6pt,pattern thickness=0.75pt,pattern radius=0pt, pattern color={rgb, 255:red, 0; green, 0; blue, 0}] (278.75,149.93) -- (305,149.93) -- (293.75,167.97) -- (267.5,167.97) -- cycle ;
%Straight Lines [id:da4427953065536345] 
\draw    (286.25,158.95) -- (284.79,87.92) ;
\draw [shift={(284.75,85.92)}, rotate = 448.82] [color={rgb, 255:red, 0; green, 0; blue, 0 }  ][line width=0.75]    (10.93,-3.29) .. controls (6.95,-1.4) and (3.31,-0.3) .. (0,0) .. controls (3.31,0.3) and (6.95,1.4) .. (10.93,3.29)   ;
%Curve Lines [id:da29204072067186404] 
\draw    (274,137) .. controls (278,131.6) and (284,131.4) .. (286,135.6) ;

% Text Node
\draw (247,81.4) node [anchor=north west][inner sep=0.75pt]  [font=\small]  {$\vec{B}$};
% Text Node
\draw (277,61.4) node [anchor=north west][inner sep=0.75pt]  [font=\small]  {$\dd \vec{S}$};
% Text Node
\draw (271,116) node [anchor=north west][inner sep=0.75pt]  [font=\small]  {$\theta $};


\end{tikzpicture}

\end{center}
    Thông lượng của từ trường $\ot{B}$ tiết diện $S$:
    %Hình 2
    $$\Phi=\int_S \ot{B}\cdot \dd \ot{S}.$$
    
    với $\dd\ot{S}=\dd S\cdot \ot{n}$ trong đó $\ot{n}$ là vector pháp tuyến đơn vị của $\dd S$.
\subsection{Dòng điện cảm ứng.}
    Khi từ thông qua một mạch kín thay đổi, trong mạch xuất hiện dòng điện cảm ứng có suất điện động:
    $$\varepsilon=-\dfrac{\dd\Phi}{\dd t}.$$
    mà suất điện động trong mạch kín $\di \varepsilon=\oint \ot{E}\cdot\dd\ot{l}$. Từ đó, rút ra được mối quan hệ giữa điện trường $\ot{E}$ và từ trường biến thiên:
    $$\oint \ot{E}\cdot \dd\ot{l}=-\int \dfrac{\partial \ot{B}}{\partial t}\cdot \dd\ot{S},$$
    Theo định lý Stokes : $$\oint \ot{E}\cdot \dd\ot{l}=\int \left(\nabla \times \ot{E}\right) \cdot \dd\ot{S},$$\\
    Ta thu được phương trình dạng vi phân của định luật Faraday:
    $$\nabla \times \ot{E}= - \dfrac{\partial \ot{B}}{\partial t}.$$
    \textbf{Lưu ý:} trong trường hợp từ trường $\ot{B}$ là không đổi, định luật Faraday trở thành $\nabla \times \ot{E}=0$ (hay $\di\oint \ot{E}\cdot \dd\ot{l}=0$).
\subsection{Tự cảm.}
    Khi trong mạch kín có dòng điện biến đổi theo thời gian làm cho từ thông bị biến thiên, trong mạch sẽ xuất hiện hiện tượng tự cảm.\\
    Từ thông gửi qua mạch tỉ lệ với dòng điện chạy trong mạch:
    $$\Phi=Li.$$
    với $L$ là hệ số tự cảm của mạch, phụ thuộc vào hình dạng, kích cỡ và môi trường xunh quanh mạch điện.\\
\subsubsection{Độ tự cảm của một ống dây.}
    Vì chiều dài của ống dây là rất lớn so với bán kính của nó nên từ trường bên trong ống dây có thể coi như là đều:
    $$B=\mu\mu_0In=\mu\mu_0I\dfrac{N}{l},$$
    với $N$ là số vòng dây, còn $l$ là chiều dài ống dây.\\
    Từ thông gửi qua tiết diện S của ống dây:
    $$\Phi=NBS=N\left(\mu\mu_0I\dfrac{N}{l}\right)S=\dfrac{\mu\mu_0N^2S}{l}I.$$
    Độ tự cảm của ống dây: $L=\dfrac{\mu\mu_0N^2S}{l}.$
\subsubsection{Suất điện động tự cảm.}
    $$\varepsilon=\dfrac{\dd\Phi}{\dd t}=\dfrac{\dd(Li)}{\dd t}.$$
\subsection{Năng lượng trong cuộn cảm.}
    $$W=\dfrac{1}{2}Li^2.$$
\subsection{Hỗ cảm.}
    \begin{center}


\tikzset{every picture/.style={line width=0.75pt}} %set default line width to 0.75pt        

\begin{tikzpicture}[x=0.7pt,y=0.7pt,yscale=-0.9,xscale=0.9]
%uncomment if require: \path (0,529); %set diagram left start at 0, and has height of 529

%Curve Lines [id:da1117818098313792] 
\draw    (148,403.33) .. controls (155.6,387.33) and (222.6,382.33) .. (275.6,386.33) .. controls (328.6,390.33) and (291.68,403.76) .. (309.6,411.33) .. controls (327.52,418.91) and (287.91,435.57) .. (237.6,437.33) .. controls (187.29,439.1) and (148.6,425.33) .. (148,403.33) -- cycle ;
%Curve Lines [id:da04941341869215332] 
\draw    (114.6,272.33) .. controls (122.2,256.33) and (161.6,272.33) .. (214.6,276.33) .. controls (267.6,280.33) and (294.68,260.76) .. (312.6,268.33) .. controls (330.52,275.91) and (296.91,314.57) .. (246.6,316.33) .. controls (196.29,318.1) and (112.6,292.33) .. (114.6,272.33) -- cycle ;
%Shape: Triangle [id:dp6309583075456229] 
\draw  [fill={rgb, 255:red, 74; green, 74; blue, 74 }  ,fill opacity=1 ] (241.44,436.64) -- (234,443.04) -- (233.47,432) -- cycle ;
%Curve Lines [id:da4196100196783519] 
\draw    (160,486.27) .. controls (193.66,445.68) and (185.18,233.57) .. (123.87,185.67) ;
\draw [shift={(122,184.27)}, rotate = 395.53999999999996] [fill={rgb, 255:red, 0; green, 0; blue, 0 }  ][line width=0.08]  [draw opacity=0] (8.93,-4.29) -- (0,0) -- (8.93,4.29) -- cycle    ;
%Curve Lines [id:da2530787268456667] 
\draw    (209,496.27) .. controls (218.85,422.39) and (223.85,221.42) .. (204.88,160.93) ;
\draw [shift={(204,158.27)}, rotate = 430.66999999999996] [fill={rgb, 255:red, 0; green, 0; blue, 0 }  ][line width=0.08]  [draw opacity=0] (8.93,-4.29) -- (0,0) -- (8.93,4.29) -- cycle    ;
%Curve Lines [id:da4587371111087095] 
\draw    (265,490.27) .. controls (251.14,430.87) and (251.98,220.53) .. (295.66,166.85) ;
\draw [shift={(297,165.27)}, rotate = 491.42] [fill={rgb, 255:red, 0; green, 0; blue, 0 }  ][line width=0.08]  [draw opacity=0] (8.93,-4.29) -- (0,0) -- (8.93,4.29) -- cycle    ;
%Curve Lines [id:da7120598464417665] 
\draw    (300,476.27) .. controls (279.21,393.11) and (313.31,288.38) .. (370.27,245.55) ;
\draw [shift={(372,244.27)}, rotate = 504.09] [fill={rgb, 255:red, 0; green, 0; blue, 0 }  ][line width=0.08]  [draw opacity=0] (8.93,-4.29) -- (0,0) -- (8.93,4.29) -- cycle    ;
%Shape: Triangle [id:dp991850232756601] 
\draw  [fill={rgb, 255:red, 74; green, 74; blue, 74 }  ,fill opacity=1 ] (288.71,304.98) -- (284.97,314.05) -- (279.51,304.44) -- cycle ;

% Text Node
\draw (230,446) node [anchor=north west][inner sep=0.75pt]  [font=\small]  {$i_{1}$};
% Text Node
\draw (158,139.67) node [anchor=north west][inner sep=0.75pt]  [font=\small]  {$\overrightarrow{B_{1}}$};
% Text Node
\draw (275,316) node [anchor=north west][inner sep=0.75pt]  [font=\small]  {$i_{2}$};
% Text Node
\draw (104,252.67) node [anchor=north west][inner sep=0.75pt]  [font=\small]  {$2$};
% Text Node
\draw (134,397.67) node [anchor=north west][inner sep=0.75pt]  [font=\small]  {$1$};


\end{tikzpicture}

    \end{center}
    Khi hai vòng dây dẫn kín (1) và (2) đặt gần nhau, các đường sức từ của từ trường $B_1$ gây bởi dòng $i_1$ ở vòng dây (1) đi qua tiết diện giới hạn bởi vòng dây (2). Thông lượng (từ thông) của từ trường $B_1$ gửi qua vòng dây (2) là:
    $$\Phi_{12}=\int \ot{B_1}\cdot \dd\ot{S_2}.$$
    Từ thông $\Phi_{12}$ tỉ lệ với dòng điện $i_1$:
    $$\Phi_{12}=M_{12}i_1,$$
    với $M_{12}$ là độ hỗ cảm của vòng dây (1) đối với vòng dây (2), phụ thuộc vào hình dạng kích thước, vị trí tương đối của các mạch điện và môi trường xunh quanh chúng.
    Tương tự, từ thông $\Phi_{21}$ do dòng điện $i_2$ ở vòng (2) gửi qua vòng (1):
    $$\Phi_{21}=M_{21}i_2.$$
    Nếu không có vật sắt từ, ta luôn có: $M_{12}=M_{21}=M.$
\subsection{Năng lượng hỗ cảm.}
    $$W=\pm \dfrac{1}{2}Mi_1i_2.$$
    Giá trị năng lượng hỗ cảm của hai mạch dương hay âm phụ thuộc vào từ thông đoạn mạch này gửi qua đoạn mạch kia là dương hay âm.
\section{Hệ phương trình Maxwell.}
 Từ những phần trước ta đã tìm hiểu về các định luật, phương trình về div và rot của điện và từ trường:
 \begin{itemize}
 	\item $\ot{\nabla}\cdot\ot{E}=\dfrac{\varrho}{\varepsilon_0} $ (Định luật Gauss),
 	\item $\ot{\nabla}\cdot\ot{B}=0$,
 	\item $\nabla \times \ot{E}= - \dfrac{\partial \ot{B}}{\partial t}$ (Định luật Faraday),
 	\item $\ot{\nabla}\times\ot{B}= \mu_0\ot{j}$ (Định luật Ampere).
 \end{itemize}
 \subsection{Định luật Ampere bổ sung bởi Maxwell.}
 	Lấy div cho cả hai vế của phương trình định luật Ampere:
 	$$\ot{\nabla}\cdot\left(\ot{\nabla}\times\ot{B}\right)= \mu_0\ot{\nabla}\cdot\ot{j},$$
 	Ta có thể dễ dàng chứng minh được vế trái của phương trình bằng không, nhưng vế phải $\ot{\nabla}\cdot\ot{j}=0$ chỉ khi dòng điện là dòng ổn định. Đối với dòng điện biến thiên thì điều này không còn đúng nữa.\\
 	Bây giờ, ta xét tới trường hợp dòng điện không ổn định:
	 $$\ot{\nabla}\cdot\ot{j}=-\dfrac{\partial\varrho}{\partial t}=-\dfrac{\partial}{\partial t}\left(\varepsilon_0\ot{\nabla}\cdot\ot{E}\right)=-\ot{\nabla}\cdot\left(\varepsilon_0\dfrac{\partial \ot{E}}{\partial t}\right),$$
	Nếu ta thay $\ot{j}$ trong phương trình định luật Ampere bằng tổng $\ot{j}+\varepsilon_0 \left(\dfrac{\partial\ot{E}}{\partial t}\right)$ thì div vế phải của phương trình sẽ thoả mãn bằng không:
	$$\ot{\nabla}\times\ot{B}=\mu_0\ot{j}+\mu_0\varepsilon_0\dfrac{\partial \ot{E}}{\partial t}.$$
	Số hạng được thêm vào trong phương trình được gọi là dòng điện dịch:
	$$\ot{j_d}=\varepsilon_0\dfrac{\partial \ot{E}}{\partial t}.$$
\subsection{Hệ phương trình Maxwell.}
	\begin{itemize}
		\item $\ot{\nabla}\cdot\ot{E}=\dfrac{\varrho}{\varepsilon_0} $ (Định luật Gauss),
		\item $\ot{\nabla}\cdot\ot{B}=0$,
		\item $\nabla \times \ot{E}= - \dfrac{\partial \ot{B}}{\partial t}$ (Định luật Faraday),
		\item $\ot{\nabla}\times\ot{B}= \mu_0\ot{j}+\mu_0\varepsilon_0\dfrac{\partial \ot{E}}{\partial t}$ (Định luật Ampere được bổ sung bởi Maxwell).
	 \end{itemize}
\subsection{Hệ phương trình Maxwell trong vật chất.}
Dòng điện gây bởi sự biến thiên của các điện tích liên kết theo thời gian:
	$$\oint\ot{j_p}\cdot\dd\ot{S}=-\dfrac{\dd }{\dd t}\int\varrho_b\dd V,$$
Áp dụng định lý Stokes:
	$$\int\left(\ot{\nabla}\cdot\ot{j_p}\right)\dd V=-\int\dfrac{\dd\varrho_b }{\dd t}\dd V,$$
Phương trình trên đúng cho mọi thể tích mà ta lấy tích phân nên:	
	$$\ot{\nabla}\cdot\ot{j_p}=-\dfrac{\dd\varrho_b }{\dd t}=-\dfrac{\dd }{\dd t}\left(-\ot{\nabla}\cdot\ot{P}\right)=\ot{\nabla}\cdot\dfrac{\dd \ot{P}}{\dd t}.$$
Mật độ dòng điện gây bởi các điện tích phân cực:	
	$$\ot{j_p}=\dfrac{\dd \ot{P}}{\dd t}.$$
Như vậy dòng điện trong vật dẫn sẽ bao gồm 3 thành phần:
	\begin{align*}
	\ot{j}&=\ot{j_f}+\ot{j_p}+\ot{j_b}\\
	&=\ot{j_f}+\dfrac{\dd \ot{P}}{\dd t}+\ot{\nabla}\times\ot{M}.
	\end{align*}
	Thế vào biểu thức định luật Ampere:
	\begin{align*}
	&\ot{\nabla}\times\ot{B}= \mu_0\left(\ot{j_f}+\dfrac{\dd \ot{P}}{\dd t}+\ot{\nabla}\times\ot{M}\right)+\mu_0\varepsilon_0\dfrac{\partial \ot{E}}{\partial t},\\
	\Rightarrow&\ot{\nabla}\times\left(\dfrac{1}{\mu_0}\ot{B}-\ot{M}\right)=\mu_0\ot{j_f}+\dfrac{\partial}{\partial t}\left(\varepsilon_0\ot{E}+\ot{P}\right).
	\end{align*}
	Từ đó ta thu được phương trình định luật Ampere trong vật chất:
	$$\ot{\nabla}\times\ot{H}=\mu_0\ot{j_f}+\dfrac{\partial\ot{D}}{\partial t}.$$
	Như vậy, hệ phương trình Maxwell bên trong vật chất:
	\begin{itemize}
		\item $\ot{\nabla}\cdot\ot{D}=\varrho_f $,
		\item $\ot{\nabla}\cdot\ot{B}=0$,
		\item $\nabla \times \ot{E}= - \dfrac{\partial \ot{B}}{\partial t}$,
		\item $\ot{\nabla}\times\ot{H}= \mu_0\ot{j_f}+\dfrac{\partial \ot{D}}{\partial t}$.
	\end{itemize}

\section{Định lý Poynting.}
\subsection{Vector Poynting.}
 Năng lượng dự trữ của điện trường:
 $$W_e=\dfrac{\varepsilon_0}{2}\int E^2\dd V. $$   
 Tương tự, năng lượng dự trữ của từ trường:
 $$W_e=\dfrac{1}{2\mu_0}\int B^2\dd V.$$
 Tổng mật độ năng lượng trên một đơn vị thể tích dự trữ trong một điện từ trường :
 $$u=\dfrac{1}{2}\left(\varepsilon_0E^2+\dfrac{1}{2\mu_0}B^2\right).$$   
 Công suất tác dụng lên các điện tích bên trong một thể tích $V$:
 $$\dfrac{\dd W}{\dd t}=\int\left(\ot{E}\cdot\ot
 {j}\right)\dd V.$$
 Áp dụng định luật Ampere (với sự bổ sung của Maxwell):
 \begin{align*}
 	\ot{E}\cdot\ot{j}&=\ot{E}\cdot\left[\dfrac{1}{\mu_0}\ot{\nabla}\times\ot{B}-\varepsilon_0\dfrac{\partial \ot{E}}{\partial t}\right],\\
 	\ot{E}\cdot\left(\ot{\nabla}\times\ot{B}\right)&=\ot{B}\cdot\left(\ot{\nabla}\times\ot{E}\right)-\ot{\nabla}\cdot\left(\ot{E}\times\ot{B}\right).\\
 \end{align*}
Áp dụng định luật Faraday, ta được:
$$\ot{E}\cdot\left(\ot{\nabla}\times\ot{B}\right)=-\ot{B}\cdot\dfrac{\partial \ot{B}}{\partial t}-\ot{\nabla}\cdot\left(\ot{E}\times\ot{B}\right).$$
Từ đó
\begin{align*}
	\ot{E}\cdot\ot{j}&=-\dfrac{\partial }{\partial t}\left(\varepsilon_0\ot{E}\cdot\dd\ot{E}+\dfrac{1}{\mu_0}\ot{B}\cdot\dd\ot{B}\right)-\dfrac{1}{\mu_0}\ot{\nabla}\cdot\left(\ot{E}\times\ot{B}\right)\\
	&=-\dfrac{1}{2}\dfrac{\partial }{\partial t}\left(\varepsilon_0 E^2+\dfrac{1}{\mu_0}B^2\right)-\dfrac{1}{\mu_0}\ot{\nabla}\cdot\left(\ot{E}\times\ot{B}\right).
	\end{align*}
Thay vào công thức và áp dụng định lý Gauss:
$$\dfrac{\dd W}{\dd t}= -\dfrac{\dd}{\dd t}\int\dfrac{1}{2}\left(\varepsilon_0 E^2+\dfrac{1}{\mu_0}B^2\right)\dd V- \dfrac{1}{\mu_0}\oint \left(\ot{E}\times\ot{B}\right)\cdot\dd \ot{S}.$$
Đây chính là biểu thức của thuyết Poynting. Trong đó, tích phân thứ nhất ở vế trái chính là năng lượng được dự trữ trong điện từ trường, tích phân thứ hai là năng lượng truyền ra khỏi thể tích $V$ trong một đơn vị thời gian. Năng lượng truyền qua một đơn vị diện tích trong một đơn vị thời gian gọi là vector Poynting:
$$\ot{S}=\dfrac{1}{\mu_0}\left(\ot{E}\times\ot{B}\right).$$
\subsection{Động lượng và moment động lượng.}
Lực điện từ tác dụng lên các điện tích chứa trong thể tích $\dd V$:
$$\dd\ot{F}=\varrho\left(\ot{E}+\ot{v}\times\ot{B}\right)\dd V.$$
Lực tác dụng lên một đơn vị thể tích là:
$$\ot{f}=\varrho\left(\ot{E}+\ot{v}\times\ot{B}\right)=\varrho\ot{E}+\ot{j}\times\ot{B}.$$
Áp dụng định luật Gauss và định luật Ampere ta thu được:
$$\ot{f}=\varepsilon_0\ot{E}\left(\ot{\nabla}\cdot\ot{E}\right)+\left(\dfrac{1}{\mu_0}\ot{\nabla}\times\ot{B}-\varepsilon_0\dfrac{\partial \ot{E}}{\partial t}\right)\times\ot{B}.$$
Bên cạnh đó,
\begin{align*}
\dfrac{\partial}{\partial t}\left(\ot{E}\times\ot{B}\right)&=\dfrac{\partial\ot{E}}{\partial t}\times\ot{B}+\ot{E}\times\dfrac{\partial\ot{B}}{\partial t},\\
&=\dfrac{\partial\ot{E}}{\partial t}\times\ot{B}+\ot{E}\times\left(-\ot{\nabla}\times\ot{E}\right).
\end{align*}
Thay lại vào biểu thức của $\ot{f}$:
\begin{align*}
\ot{f}&=\varepsilon_0\ot{E}\left(\ot{\nabla}\cdot\ot{E}\right)-\dfrac{1}{\mu_0}\times\ot{B}\left(\ot{\nabla}\times\ot{B}\right)-\varepsilon_0\ot{E}\times\left(\ot{\nabla}\times\ot{E}\right)-\varepsilon_0\dfrac{\partial}{\partial t}\left(\ot{E}\times\ot{B}\right).
\end{align*}
Áp dụng quy tắc div của một tích vô hướng hai vector:
\begin{align*}
\ot{\nabla}\left(E^2\right)=2\left(\ot{E}\cdot\ot{\nabla}\right)\ot{E}+2\ot{E}\times\left(\ot{\nabla}\times\ot{E}\right),\\
\ot{\nabla}\left(B^2\right)=2\left(\ot{B}\cdot\ot{\nabla}\right)\ot{E}+2\ot{B}\times\left(\ot{\nabla}\times\ot{B}\right).
\end{align*}
Từ đó,
\begin{align*}
	\ot{f}=\varepsilon_0\left[\left(\ot{\nabla}\cdot\ot{E}\right)\ot{E}+\left(\ot{E}\cdot\ot{\nabla}\right)\ot{E}\right]&+\dfrac{1}{\mu_0}\left[\left(\ot{\nabla}\cdot\ot{B}\right)\ot{B}+\left(\ot{B}\cdot\ot{\nabla}\right)\ot{B}\right]\\
	&-\dfrac{1}{2}\ot{\nabla}\left(\varepsilon_0E^2+\dfrac{1}{\mu_0}B^2\right)-\varepsilon_0\dfrac{\partial}{\partial t}\left(\ot{E}\times\ot{B}\right),
\end{align*}
vì trong hệ phương trình Maxwell $\left(\ot{\nabla}\cdot\ot{B}\right)\ot{B}=0$.

Định lý biến thiên động lượng:
\begin{align*}
	\dfrac{\dd \ot{p}}{\dd t}=\ot{F}=\int\ot{f}\dd V.
\end{align*}
Sử dụng công thức lực trên một đơn vị thể tích vừa thu được:
\begin{align*}
	\dfrac{\dd \ot{p}}{\dd t}=\int\varepsilon_0\left[\left(\ot{\nabla}\cdot\ot{E}\right)\ot{E}+\left(\ot{E}\cdot\ot{\nabla}\right)\ot{E}\right]&\dd V+\dfrac{1}{\mu_0}\int\left[\left(\ot{\nabla}\cdot\ot{B}\right)\ot{B}+\left(\ot{B}\cdot\ot{\nabla}\right)\ot{B}\right]\dd V\\
	&-\dfrac{1}{2}\int{\nabla}\left(\varepsilon_0E^2+\dfrac{1}{\mu_0}B^2\right)\dd V-\varepsilon_0\int\dfrac{\partial}{\partial t}\left(\ot{E}\times\ot{B}\right)\dd V.
\end{align*}
Số hạng cuối cùng là độ biến thiên động lượng của điện từ trường:
\begin{align*}
	\dfrac{\dd \ot{p}_{field}}{\dd t}&=\varepsilon_0\int\dfrac{\partial}{\partial t}\left(\ot{E}\times\ot{B}\right)\dd V,\\
	\Rightarrow\ot{p}_{field}&=\varepsilon_0\int\left(\ot{E}\times\ot{B}\right)\dd V,\\
	&=\varepsilon_0\mu_0\int\ot{S_p}\dd V.
\end{align*}
Mật độ động lượng trên một đơn vị thể tích:
$$\ot{g}=\varepsilon_0\left(\ot{E}\times\ot{B}\right)=\varepsilon_0\mu_0\ot{S_p}.$$
Đồng thời ta cũng thu được công thức mật độ moment động lượng:
$$\ot{l}=\ot{r}\times\ot{g}=\varepsilon_0\left[\ot{r}\times\left(\ot{E}\times\ot{B}\right)\right].$$
\section{Điện từ trường tương đối tính.}
Mối quan hệ giữa điện từ trường trong hệ quy chiếu $S'$ chuyển động với vận tốc $\ot{v}$ và hệ quy chiếu đứng yên $S$:
\begin{align*}
	\ot{E}'_{\bot}&=\gamma\left[\ot{E}_{\bot}+\ot{v}\times\ot{B}_{\bot}\right],\\
	\ot{E}'_{\parallel}&=\ot{E}_{\parallel},\\
	\ot{B}'_{\bot}&=\gamma\left[\ot{B}_{\bot}-\dfrac{\ot{v}}{c^2}\times\ot{B}_{\bot}\right],\\
	\ot{B}'_{\parallel}&=\ot{B}_{\parallel} \text{ với } \gamma=\dfrac{1}{\sqrt{1-\dfrac{v^2}{c^2}}}.
\end{align*}
trong đó $\ot{E}_{\parallel}$, $\ot{B}_{\parallel}$ là thành phần điện từ trường song song với phương của $\ot{v}$, còn $\ot{E}_{\bot}$, $\ot{B}_{\bot}$ là thành phần điện từ trường vuông góc với phương của $\ot{v}$.









% cần sửa
\section{Hệ đơn vị Gaussian}


\begin{center}
\begin{longtable}{|l|c|c|} %{|p{5cm}|p{6cm}|p{6cm}|} 
\caption{Bảng so sánh giữa hai hệ.} \label{tab:long} \\

\hline \multicolumn{1}{|c|}{\textbf{}} & \multicolumn{1}{c|}{\textbf{SI}} & \multicolumn{1}{c|}{\textbf{Gaussian}} \\ \hline 
\endfirsthead

\multicolumn{3}{c}%
{{\bfseries \tablename\ \thetable{} -- Bảng so sánh giữa hai hệ.}} \\
\hline \multicolumn{1}{|c|}{\textbf{}} & \multicolumn{1}{c|}{\textbf{SI}} & \multicolumn{1}{c|}{\textbf{Gaussian}} \\ \hline 
\endhead

%\hline \multicolumn{3}{|r|}{{Continued on next page}} \\ \hline
%\endfoot

%\hline \hline
%\endlastfoot


\hline
 \text{Lực tương tác tĩnh điện}  &  $ \ot{F}=\dfrac{1}{4\pi\varepsilon_0}\dfrac{qQ}{\left|\ot{r}-\ot{r'}\right|^3}\cdot\left(\ot{r}-\ot{r'}\right) $  & 
  $\ot{F}= \dfrac{qQ}{\left|\ot{r}-\ot{r'}\right|^3}\cdot\left(\ot{r}-\ot{r'}\right)$ \\[6pt]

\hline
\text{Cường độ điện trường} &  $\ot{E}=\dfrac{1}{4\pi\varepsilon_0}\dfrac{q}{\left|\ot{r}-\ot{r'}\right|^3}\cdot\left(\ot{r}-\ot{r'}\right)$  &  $\ot{E}=\dfrac{q}{\left|\ot{r}-\ot{r'}\right|^3}\cdot\left(\ot{r}-\ot{r'}\right)$ \\ 

        %\hline
	  %\text{Điện thông} &  $\Phi=\int \ot{E}\cdot \dd \ot{S}$  & \\
	  
	 \hline
	   \text{Thế năng tĩnh điện}  & $ V=\dfrac{1}{4\pi\varepsilon_0}\dfrac{q_1q_2}{r_{12}}$  &  $ V=\dfrac{q_1q_2}{r_{12}} $\\
	   
	 %\hline
	  %\text{Lưỡng cực điện} & $\ot{p}=q\ot{l}$  & \\
	  
	 \hline
	  \text{Thế năng lưỡng cực điện}&  $\varphi=\dfrac{1}{4\pi\varepsilon_0}\dfrac{\ot{p}\cdot\ot{r}}{r^3}$  &  $\varphi=\dfrac{\ot{p}\cdot\ot{r}}{r^3}$  \\
	  
	 \hline
	  \text{Độ phân cực điện} & $\ot{P} = \varepsilon_0\chi_e \ot{E}$  &  $\ot{P} =\chi_e\ot{E}$  \\
	  
	 \hline
	 \text{Hằng số điện môi} & $\varepsilon = 1+\chi_e$  &  $\varepsilon = 1+4\pi\chi_e$  \\
	 
	 \hline
	  \text{Tụ điện phẳng} &  $C=\dfrac{\varepsilon\varepsilon_0S}{d}$  &  $C=\dfrac{\varepsilon S}{4\pi d}$  \\
	  
	 \hline
	 \text{Tụ điện cầu}& $C=\dfrac{4\pi\varepsilon\varepsilon_0R_1R_2}{R_2-R_1}$& $C=\dfrac{\varepsilon R_1 R_2}{R_2-R_1}$\\
	 
	 \hline
	 \text{Tụ điện trụ}& $C=\dfrac{2\pi\varepsilon\varepsilon_0l}{\ln\left(\dfrac{R_2}{R_1}\right)}$& $C=\dfrac{\varepsilon l}{2\ln\left(\dfrac{R_2}{R_1}\right)}$\\
	 
	 \hline
	 \text{Từ trường}&$\ot{B}=\dfrac{\mu_0}{4\pi}\int\dfrac{I\dd\ot{l'}\times\left(\ot{r}-\ot{r'}\right)}{\left|\ot{r}-\ot{r'}\right|^3} $& $\ot{B}=\dfrac{1}{c}\int\dfrac{I\dd\ot{l'}\times\left(\ot{r}-\ot{r'}\right)}{\left|\ot{r}-\ot{r'}\right|^3}$\\
	 \hline
	 \text{Lực Lorentz}&$\ot{F}=q\ot{v}\times\ot{B} $& $\ot{F}=\dfrac{q}{c}\ot{v}\times\ot{B}$\\
	 %\hline
	 %\text{Lực từ}& & ,\\
	 \hline
	 \text{Lưỡng cực từ}& $\ot{p_m}=I\ot{S}$ & $\ot{p_m}=\dfrac{I\ot{S}}{c}$\\
	 \hline
	 \text{Vector từ hoá}& $\ot{M}=\chi_m\dfrac{\ot{B}}{\mu_0}$& $\ot{M}=\chi_m\ot{B}$ \\
	 \hline
	 \text{Độ từ thẩm}& $\mu =1+\chi_m $& $\mu =1+4\pi\chi_m $\\
	 \hline
	 \text{Cường độ từ trường}& $\ot{H}=\dfrac{\ot{B}}{\mu\mu_0}$ & $\ot{H}=\dfrac{\ot{B}}{\mu}$\\
	 %\hline
	 %\text{Từ thông}& $\Phi=\int \ot{B}\cdot \dd \ot{S}$ & \\
	 \hline
	 \text{Suất điện động cảm ứng}& $\varepsilon=-\dfrac{\dd\Phi}{\dd t}$& $\varepsilon=-\dfrac{1}{c}\dfrac{\dd\Phi}{\dd t}$\\
	 %\hline
	 %\text{Suất điện động tự cảm}&$\varepsilon=L\dfrac{\dd i}{\dd t} $& \\
	 \hline
	 \text{Độ tự cảm của ống dây}& $L=\dfrac{\mu\mu_0N^2S}{l} $&$\dfrac{4\pi\mu N^2S}{c^2l}$ ,\\
	 %\hline
	 %\text{Suất điện động hỗ cảm}&$\varepsilon_{12}=-M_{12}\dfrac{\dd i_1}{\dd t}$ & \\
	 %\hline
	 %\text{Hệ phương trình Maxwell}&
	 %$\begin{aligned}
	 %    \ot{\nabla}\cdot\ot{E}&=\dfrac{\varrho}{\varepsilon_0}\\
	 %    \ot{\nabla}\cdot\ot{B}&=0,\\
	%	\nabla \times \ot{E}&= - \dfrac{\partial \ot{B}}{\partial t},\\
	%	\nabla \times \ot{E}&= - \dfrac{\partial \ot{B}}{\partial t},\\
	%	\ot{\nabla}\times\ot{B}&= \mu_0\ot{j}+\mu_0\varepsilon_0\dfrac{\partial \ot{E}}{\partial t}.
	 %\end{aligned}$&\\
	 \hline
	 \text{Năng lượng của điện từ trường}&$u=\dfrac{1}{2}\left(\varepsilon_0E^2+\dfrac{1}{2\mu_0}B^2\right)$ & $u=\dfrac{1}{8\pi}\left(E^2+B^2\right)$\\
	 \hline
	 \text{Vector Poynting}& $\ot{S}=\dfrac{1}{\mu_0}\left(\ot{E}\times\ot{B}\right)$& $\ot{S}=\dfrac{c}{3\pi\mu_0}\left(\ot{E}\times\ot{B}\right)$\\
	 \hline
\end{longtable}
\textbf{Tất cả các bài toán dùng hệ dơn vị Gauss sẽ được đánh dấu bằng kí hiệu (G) bên cạnh tên bài.}

\end{center}
\end{appendices}