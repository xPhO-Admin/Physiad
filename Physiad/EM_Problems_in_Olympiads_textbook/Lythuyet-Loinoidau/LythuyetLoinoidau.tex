\chapter*{Lời giới thiệu}
\addcontentsline{toc}{chapter}{Lời giới thiệu}
Thời gian gần đây, theo xu hướng phát triển của công nghệ và kĩ thuật điện tử, phần điện từ học cũng chiếm một tỉ lệ điểm rất lớn trong các đề thi Olympic Vật Lí với các nội dung ngày càng khó và cập nhật. Tuy nhiên,
các nguồn tư liệu hiện thời về điện từ học bằng tiếng Việt chưa thật sự phổ biến và còn khó tiếp cận. Thấu hiểu điều đó, nhóm tác giả \textsc{Physiad} bao gồm các học sinh đã từng tham gia nhiều kì thi quốc gia, quốc tế và cũng từng gặp khó khăn khi tìm các tài liệu tham khảo đã quyết định xuất bản cuốn ``Tuyển tập các bài tập điện từ trong các kỳ thi Olympic'' với $176$ bài tập điện từ có lời giải được tuyển tập và sưu tầm từ những kì thi Olympic Vật Lí uy tín nhất. Dựa vào số lượng bài tập trong mỗi chương, các bạn có thể đánh giá được tần suất xuất hiện của các dạng câu hỏi, từ đó lập ra chiến thuật ôn luyện tối ưu. Các bài tập đều được dịch sang tiếng Việt với ngôn ngữ quen thuộc, dễ hiểu nhưng vẫn đảm bảo tính đúng đắn về mặt khoa học. Ngoài ra, các bài tập trong mỗi chương còn được sắp xếp theo thứ tự từ dễ đến khó kèm theo một chương lý thuyết tổng hợp tất cả các kiến thức vật lí và toán học nền tảng cần thiết trong quá trình sử dụng cuốn sách.\\

Với các mức độ bài tập đa dạng, chúng tôi hi vọng cuốn sách sẽ là nguồn tài liệu tham khảo hữu ích cho các bạn học sinh đang chuẩn bị cho các kì thi Vật Lí, từ cấp trường cho đến khu vực và quốc tế. Để sử dụng sách hiệu quả nhất, các bạn học sinh hãy cố gắng tự giải các bài tập và chỉ dùng phần đáp án để đối chiếu với các kết quả mình thu được.\\
Là cuốn sách đầu tay của một nhóm tác giả chưa có nhiều kinh nghiệm, những sai sót và lỗi trong quá trình biên soạn là không thể tránh khỏi, vì vậy chúng tôi rất mong nhận được những ý kiến đóng góp từ các bạn học sinh và các thầy cô trên cả nước. Chúng tôi tin rằng với sự đồng hành của các bạn cùng sức trẻ và nhiệt huyết sẵn có của nhóm tác giả, cuốn sách này cùng các sản phẩm sau của nhóm sẽ ngày càng được hoàn thiện để phần nào đó có thể giúp đỡ các bạn học sinh trong quá trình chinh phục vật lí nói riềng và khoa học nói chung.
\begin{flushright}
\large\textsc{Nhóm Physiad}
\end{flushright}
